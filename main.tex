\documentclass{article}

\usepackage{amssymb}
\usepackage[T1]{fontenc}
\usepackage[latin1]{inputenc}


\newcommand{\assert}{\texttt{assert}}
\newcommand{\assume}{\texttt{assume}}
\newcommand{\baseaddr}{\texttt{{\textbackslash}baseaddr}}
\newcommand{\boundseparated}{\texttt{{\textbackslash}bound\_separated}}
\newcommand{\Exists}{\texttt{{\textbackslash}exists}~}
\newcommand{\Forall}{\texttt{{\textbackslash}forall}~}
\newcommand{\freed}{\texttt{{\textbackslash}freed}}
\newcommand{\fresh}{\texttt{{\textbackslash}fresh}}
\newcommand{\fullseparated}{\texttt{{\textbackslash}full\_separated}}
\newcommand{\Max}{\texttt{max}}
\newcommand{\nothing}{\texttt{{\textbackslash}nothing}}
\newcommand{\numof}{\texttt{num\_of}}
\newcommand{\offsetmin}{\texttt{{\textbackslash}offset\_min}}
\newcommand{\offsetmax}{\texttt{{\textbackslash}offset\_max}}
\newcommand{\old}{\texttt{{\textbackslash}old}}
\newcommand{\at}{\texttt{{\textbackslash}at}}
\newcommand{\If}{\texttt{if}~}
\newcommand{\Then}{~\texttt{then}~}
\newcommand{\Else}{~\texttt{else}~}
\newcommand{\Let}{\texttt{let}~}
\newcommand{\In}{~\texttt{in}~}
\newcommand{\result}{\texttt{{\textbackslash}result}}
\newcommand{\separated}{\texttt{{\textbackslash}separated}}
\newcommand{\sizeof}{\texttt{sizeof}}
\newcommand{\strlen}{\texttt{{\textbackslash}str\_len}}
\newcommand{\Sum}{\texttt{sum}}
\newcommand{\valid}{\texttt{{\textbackslash}valid}}
\newcommand{\validrange}{\texttt{{\textbackslash}valid\_range}}

\newcommand{\N}{\ensuremath{\mathbb{N}}}
\newcommand{\ra}{\ensuremath{\rightarrow}}

\begin{document}

\emph{pure} : side-effect free

\emph{behavior}: pair pre- and post-condition, assigns

Proposition de Claude: ce document doit etre illustr� par des
exemples. Toute construction qui ne serait pas illustr�e par un
exemple sera non-retenue

\section{Frama-C base level}

\begin{itemize}
\item logical expressions = pure C expressions, without function
  calls. always defined (2-valued logic). With additional constructs:
  \begin{itemize}
  \item values in preceedings states: \old(e) and \at(e,label).

    construct \old{} is only valid in ensures clauses of function
    specification. For loop invariants, one need to put some label and
    use the \at{} construct.
  \item returned value: \result

    valid only in ensures clauses of function
    specification.

  \item quantifications:  $\Forall \tau x; e$ and $\Exists \tau x; e$
  \item \verb|==>|, \verb|<==>|, 

    Open question: \verb|=>| and \verb|<=>| instead of \verb|==>|,
    \verb|<==>| ? 

  \item $\Let \tau x = e \In e$ and $\If e \Then e \Else e$

  \item nommage de formules, de termes

  \end{itemize}

on veut la logique equationnelle classique, cad que l'axiome 
\[
\forall x, x=x
\]
est valide. Donc on ne peut pas introduire de construction
non-d�terministe comme  $(\texttt{any} x \mid P)$


\item C function behaviors
\item statement annotations:
  \begin{itemize}
  \item \assert, \assume, label
  \item loop invariants (+loop assigns ?)
  \item block behaviors (normal termination only, no goto going
    inside, no goto going outside)
  \item proposition de Patrick: permettre de poser une assertion sur
    la sortie de fonction quelque soit cette sortie. Doit pourvoir
    parler des parametres, et eventuellement des variables locales
    declarees en debut de fonction.
  \end{itemize}
\end{itemize}

Grammar:

\begin{eqnarray*}
  fun-spec &::=& (requires-clause | fun-behavior ) * variant-clause ? \\
  requires-clause &::=& "requires" predicate \\
  variant-clause &::=& "variant" term ("for" ident)? \\
  fun-behavior &::=& "behavior" ident ":"
             (assumes-clause | assigns-clause |
              ensures-clause | invariant-clause) * \\
  assumes-clause &::=& "assumes" predicate \\
  assigns-clause &::=& "assigns" (location (, location) * |
  "\nothing") \\
  ensures-clause &::=& "ensures" predicate \\
  invariant-clause &::=& "invariant" predicate \\
  location &::=& (see below)
\end{eqnarray*}

semantics:

\begin{verbatim}
  requires R_j
  behavior x_i: assumes A_i ensures E_i invariant I_i


 pre-condition : R_1 and R_2 and ...
 post-condition : (\old(A_1 and I_1) implies E_1 and I_1) and ...
\end{verbatim}


Proposition de Yannick: des requires dans les behaviors:
la semantique consiste a rajouter $(A_i \ra R_i)$ en conjonction de la
precondition globale 




Questions: "reads" clause ?

\subsection{syntax of assigns clauses}
\label{sec:locations}

\begin{verbatim}
  location ::=
       var | location->id | location.id | *location
       | location[term ? .. term?] | location[term]
       |  (\Forall \(\tau\)) | (\Forall \(\tau\) x ; predicate)
       |  location + [ term? .. term? ] | location + term
\end{verbatim}

semantics:
\begin{itemize}
\item a "location" denotes a set of terms.
\item only location denoting a set of lvalues are valid as argument to
  "assigns" clauses : constructs in the two first lines above denotes
  lvalues, but not the two last lines
\end{itemize}

example:

\begin{verbatim}
  assigns  (\Forall struct list *p ; reachable(root,p)) -> hd
\end{verbatim}

\section{Constructions for valid pointer dereferencing, separation}

\subsection{Memory blocks: physical adressing}

\verb|\base_addr|
\verb|\blocklength|


\subsection{Memory blocks: logical adressing}

a \emph{valid} pointer is a pointer such that *p is properly
allocated.

\offsetmin, \offsetmax : $\alpha * \ra \N$

\verb|\offset_min(p)| : the minimum integer i such that (p+i) is a valid
pointer

\verb|\offset_max(p)|: the maximum integer i such that (p+i) is a valid
pointer

properties:

\[
\offsetmin(p+i) = \offsetmin(p)-i
\offsetmax(p+i) = \offsetmax(p)-i
\]

syntactic sugar:
\[
\validrange(p,i,j) := \offsetmin(p) <= i \land \offsetmax(p) >= j
\valid(p) := \validrange(p,0,0)
\]

\subsection{Strings}

\strlen

Peut disparaitre

\subsection{Separation}

pointer separation :
\[
p \not\equiv q := \baseaddr(p) \neq \baseaddr(q) \lor |(char*)p - (char*)q| \geq \max(\sizeof(p),\sizeof(q))
\]

$\separated(p1,..,p_n)$ : means that for if $i<>j, p_i \not\equiv p_j$

cette construction peut-etre etendue au cas ou les $p_i$ sont des
ensembles de locations memoires, telles que definies a la section
\ref{sec:locations}. 


\section{Termination}

termination: loop variants, function variants for recursive
  functions, for integer measures.

More general case of measures on other types: use the keyword for:

\begin{eqnarray*}
&& "variant" e "for" id
\end{eqnarray*}
where $e$ has type $\tau$ and $id$ is a logic predicate of type $\tau,\tau \ra
prop$


\section{Logic specifications}

\begin{itemize}
\item logic specifications: introduction of new logic types,
  constants, functions, predicates and axioms
\item polymorphism: for example
\begin{verbatim}
type 'a list
\end{verbatim}
\item higher-order functions: $\lambda$, \Sum, \Max, \numof
\item {\tt enum}
\item recursive types, recursive definitions
\item hybrid functions and predicates: take both C types and logic
  types as arguments.
\item model variables
\end{itemize}

\section{Allocation and deallocation}

\fresh, \freed

\section{Abnormal termination}

\begin{itemize}
\item C function \verb|exit(n)|. special behavior notation
  \begin{verbatim}
    exit_behavior
       assumes
       assigns
       ensures ...
  \end{verbatim}
    where in ensures clause, \result is bound to n
\end{itemize}


\section{Dependencies information}

\begin{verbatim}
assigns Locs from Locs
\end{verbatim}

\section{Functional expressions}

trouver une syntaxe pour introduire les noms des fonctions implicites
derriere la construction
\begin{verbatim}
assigns Locs from Locs
\end{verbatim}

\section{Data invariants}


\begin{itemize}
\item Global invariants: invariants on global variables
\item Type invariants: invariants on struct, union, typedef
\end{itemize}

semantics:

an invariant is true when ... \cite{boogie}

\section{Ghost variables and statements}

\section{Annotations: in comments or in another file}

in JML, parsing is done by just removing the $/*@$ and $*/$. this can
completely modify the semantics: Example:
\begin{verbatim}
return x /*@ +1 */ ;
\end{verbatim}

The current option is to isolate the comments in a first step, and
then parse a second time. But the same problem may appear:
\begin{verbatim}
if (c) //@ assert P
   c=1;
\end{verbatim}
for the analyser, instruction \verb|c=1| is executed unconditionnaly. 



\section{Module constructions}

how to encapsulate several functions...

\section{References}
\bibliographystyle{plain}
\bibliography{biblio}

\end{document}

%%% Local Variables:
%%% mode: latex
%%% TeX-master: t
%%% End:
