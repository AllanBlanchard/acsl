\documentclass[a4paper]{report}

\usepackage{fullpage}
\usepackage{amssymb}
\usepackage[T1]{fontenc}
\usepackage{times}
\usepackage{pst-node}
\usepackage{pstcol}

\usepackage{xspace}
\usepackage{makeidx}
\makeindex

\newcommand{\experimental}{\textsc{Experimental}}

\newsavebox{\fmbox}
\newenvironment{cadre}
     {\begin{lrbox}{\fmbox}\begin{minipage}{0.98\textwidth}}
     {\end{minipage}\end{lrbox}\fbox{\usebox{\fmbox}}}

\newenvironment{todo}{\begin{quote} 
    \begin{tabular}{||p{0.8\textwidth}}
TODO~:\itshape}{\end{tabular}\end{quote}}

\newenvironment{remark}[1]{\begin{quote}\itshape 
    \begin{tabular}{||p{0.8\textwidth}}
Remarque de {#1}~:}{\end{tabular}\end{quote}}

\newenvironment{oldremark}[1]{\begin{quote}\itshape 
    \begin{tabular}{||p{0.8\textwidth}}
Vieille remarque de {#1}~:
}{
\end{tabular}\end{quote}
}

\newcommand{\keyword}[1]{\texttt{#1}\xspace}

\newcommand{\integer}{\keyword{integer}}
\newcommand{\real}{\keyword{real}}
\newcommand{\boolean}{\keyword{boolean}}
\newcommand{\assert}{\keyword{assert}}
\newcommand{\assume}{\keyword{assume}}
\newcommand{\requires}{\keyword{requires}}
\newcommand{\ensures}{\keyword{ensures}}
\newcommand{\assumes}{\keyword{assumes}}
\newcommand{\assigns}{\keyword{assigns}}
\newcommand{\reads}{\keyword{reads}}
\newcommand{\decreases}{\keyword{decreases}}
\newcommand{\boundseparated}{\keyword{{\textbackslash}bound\_separated}}
\newcommand{\Exists}{\keyword{{\textbackslash}exists}~}
\newcommand{\Forall}{\keyword{{\textbackslash}forall}~}
\newcommand{\freed}{\keyword{{\textbackslash}freed}}
\newcommand{\fresh}{\keyword{{\textbackslash}fresh}}
\newcommand{\fullseparated}{\keyword{{\textbackslash}full\_separated}}
\newcommand{\Max}{\keyword{max}}
\newcommand{\nothing}{\keyword{{\textbackslash}nothing}}
\newcommand{\numof}{\keyword{num\_of}}
\newcommand{\offsetmin}{\keyword{{\textbackslash}offset\_min}}
\newcommand{\offsetmax}{\keyword{{\textbackslash}offset\_max}}
\newcommand{\old}{\keyword{{\textbackslash}old}}
\newcommand{\at}{\keyword{{\textbackslash}at}}

\newcommand{\If}{\keyword{if}~}
\newcommand{\Then}{~\keyword{then}~}
\newcommand{\Else}{~\keyword{else}~}
\newcommand{\For}{\keyword{for}~}
\newcommand{\While}{~\keyword{while}~}
\newcommand{\Do}{~\keyword{do}~}
\newcommand{\Let}{\keyword{\textbackslash{}let}~}

\newcommand{\struct}{\keyword{struct}}
\newcommand{\union}{\keyword{union}}
\newcommand{\inter}{\keyword{inter}}
\newcommand{\typedef}{\keyword{typedef}}
\newcommand{\result}{\keyword{{\textbackslash}result}}
\newcommand{\separated}{\keyword{{\textbackslash}separated}}
\newcommand{\sizeof}{\keyword{sizeof}}
\newcommand{\strlen}{\keyword{{\textbackslash}strlen}}
\newcommand{\Sum}{\keyword{sum}}
\newcommand{\valid}{\keyword{{\textbackslash}valid}}
\newcommand{\validrange}{\keyword{{\textbackslash}valid\_range}}
\newcommand{\offset}{\keyword{{\textbackslash}offset}}
\newcommand{\blocklength}{\keyword{{\textbackslash}block\_length}}
\newcommand{\baseaddr}{\keyword{{\textbackslash}base\_addr}}
\newcommand{\comparable}{\keyword{{\textbackslash}comparable}}

\newcommand{\N}{\ensuremath{\mathbb{N}}}
\newcommand{\ra}{\ensuremath{\rightarrow}}


% BNF grammar

\newif\ifspace
\newif\ifnewentry
\newcommand{\addspace}{\ifspace \; \spacefalse \fi}
\newcommand{\term}[1]{\addspace\hbox{\texttt{#1}} \spacetrue}
\newcommand{\nonterm}[1]{%
\addspace\hbox{\textsl{#1}\ifnewentry\index{#1@\textsl{#1}!non-terminal}\fi}\spacetrue}
\newcommand{\repetstar}{^*\spacetrue}
\newcommand{\repetplus}{^+\spacetrue}
\newcommand{\repetone}{^?\spacetrue}
\newcommand{\lparen}{\addspace(}
\newcommand{\rparen}{)}
\newcommand{\orelse}{\addspace\mid\spacetrue}
\newcommand{\sep}{ \\[2mm] \spacefalse\newentrytrue}
\newcommand{\newl}{ \\ & & \spacefalse}
\newcommand{\alt}{ \\ & \mid & \spacefalse}
\newcommand{\is}{ & ::= & \newentryfalse}
\newenvironment{syntax}{$$\begin{array}{rrll}\spacefalse}{\end{array}$$}
\newcommand{\synt}[1]{$\spacefalse#1$}
\newcommand{\emptystring}{\epsilon}
\newcommand{\below}{See\; below}

% colors

\definecolor{darkgreen}{rgb}{0, 0.5, 0}

% theorems

\newtheorem{example}{Example}[chapter]

%%% Local Variables:
%%% mode: latex
%%% TeX-PDF-mode: t
%%% TeX-master: "main"
%%% End:


\begin{document}

\title{ACSL: ANSI C Specification Language}

\author{P.~Baudin \and J.-C.~Filli\^atre \and Th.~Hubert \and
  C.~March\'e \and B.~Monate \and Y.~Moy \and V.~Prevosto}

\maketitle

\chapter{Introduction}

This document is a reference manual for ACSL, an acronym for ``ANSI C
Specification Language''. This is a specification language proposed by
the CAT
project\footnote{\url{http://www.rntl.org/projet/resume2005/cat.htm},
  \todo{referer a une page publique en anglais dont la creation a ete
  decide}}, implemented in the \textsc{Frama-C} platform, which aims at
specifying behavioral properties of C source code. The main
inspiration for this language comes from the specification language of
the \textsc{Caduceus} tool~\cite{filliatre04icfem,filliatre07cav} for
deductive verification of behavioral properties of C programs. It is
itself inspired from the \emph{Java Modeling Language} (JML) which
aims at similar goals for Java source code: indeed it aims both at
\emph{runtime assertion checking} and \emph{static verification} using
the \textsc{ESC/Java2} tool~\cite{ESCJava2}, where we aim at 
\emph{static verification} and \emph{deductive verification}.

Going back further in history, JML design was guided by the general
\emph{design-by-contract} principle proposed by Bertrand Meyer, who
took his own inspiration from the concepts of preconditions and
postconditions on a routine, going back at least to Dijkstra, Floyd and
Hoare in the late 60's and early 70's, and originally implemented in
the \textsc{Eiffel} language.

In this document, we assume that the reader has a good knowledge of
the ANSI C programming language~\cite{KR88}.

\section{Organization of this document}

In this preliminary chapter we introduce some definitions and
vocabulary, and discuss generalities about this specification language.

Chapter~\ref{chap:base} presents the specification language itself.

Chapter~\ref{chap:lib} presents additional informations about
\emph{libraries} of specifications.

A detailed table of contents is given on page~\pageref{chap:contents}

\subsection*{Experimental features}

Some features in this document are considered \emph{experimental},
meaning that their syntax and semantics is still under discussion.
These features are marked with \experimental.

They must also be considered as ``advanced features'' which are not
supposed to be useful for a basic use of that specification language.


\section{Glossary}

\begin{description}
\item[pure expressions] \index{pure expression} A C expression is
  called \emph{pure} if it has no side-effect: no assignments, no
  incrementation operator \verb|++| or \verb|--|. Moreover it should
  not contain any function call, even if the called function has no
  side-effect itself.

\item[memory locations and left-values] \index{memory location}
  \index{left value}\index{l-value} A \emph{memory location} is an
  expression which denotes some place in the memory during program
  execution, either on the stack, on the heap, or in the static data
  segment. It can be either a variable identifier or an expression of
  the form $*e$, $e[e]$, $e\verb|.|id$ or $e\verb|->|id$.

  A \emph{left value}, or \emph{l-value} for short, is a memory
  location allowed in the left part of an assignment, that is a
  non-constant memory location.
 
\item[pre-state and post-state]
    \index{pre-state}\index{post-state}
    
    For a given function call, the \emph{pre-state} denotes the
    program's state at the beginning of the call, including the
    current values for the function parameters. the \emph{post-state}
    denotes the program's state at the return of the call.

\item[function behavior] \index{function behavior} \index{behavior}

  A \emph{function behavior} is a set of properties relating the
  pre-state and the post-state for any call to that function.

\item[contract] \index{contract} A \emph{contract} forms a
  specification of a function, consisting of the combination of a
  precondition (an assumption about the pre-state) and a collection
  of behaviors.

\end{description}

\oldremark{Claude}{ce document doit etre illustr� par des
exemples. Toute construction qui ne serait pas illustr�e par un
exemple sera non-retenue.}

\oldremark{Patrick}{utiliser en priorit� les constructions de JML,
ensuite les g�n�raliser lorsque cela a un sens, et en dernier recours,
en cr�er de nouvelles. Il y a peut-�tre des constructions
du C++ ou C\# � utiliser ou � g�n�raliser.
Idem avec les extensions de GCC.}

\section{Generalities about Annotations}

In this document, we consider that annotations are given as comments
written directly into C source files, so
that source files remain compilable\footnote{Other means of attaching
  annotations to source files, without modifying them, are left to user
  tools.}. Those comments must start by \verb|/*@| or \verb|//@|.

\subsection{Kinds of annotations}

\begin{itemize}
\item Global annotations:
  \begin{itemize}
  \item function contract. Such an annotation is inserted just before
    the declaration or the definition of a function.
    See section~\ref{sec:fn-behavior}.

  \item global invariant. This is allowed at the level of global declarations.
    See section~\ref{sec:invariants}.

  \item type invariant. This allows both structure or union
    invariants, and invariants on type names introduced by \typedef.
    See section~\ref{sec:invariants}.

  \item logic specifications: logic type introduction, introduction
    or definition of logic functions or predicates, axioms. Such an
    annotation is placed at the global declarations level.

  \end{itemize}

\item Statement annotations:
  \begin{itemize}
  \item \assert clause, logic label. These are allowed
    everywhere a C label is allowed, or exceptionally just before a
    block closing brace.  
    \oldremark{Claude}{About \assume{} clauses: current
      discussion is that it is not considered as an element of
      specification, so not present here. This should be part of proof
      management done by tools.} 
    \oldremark{Patrick}{En C, seules les
      instructions peuvent �tre �tiquett�es. GCC �tend cela aux
      accolades fermantes des blocs, y compris celle fermant le corps
      de fonction. On peut dire que l'on fait de m�me lorsque l'on
      place des annotations juste avant la fermerture d'un bloc.}
    
  \item loop annotation (invariant, variant, assign clauses) is
    allowed immediately before a loop statement: \For, \While,
    \Do\ldots \While. See Section~\ref{sec:loop_annot}

  \item statement contract. Very similar to a function contract.
    Semantical condition must be checked (normal termination only, no
    goto going inside, no goto going outside).  See
    Section~\ref{sec:statement_contract}
    \oldremark{Patrick}{as-t'on droit au \old dans le \ensures de
      cette annotation~? Oui, pour refer a l'etat avant le statement
      consider\'e}
    
  \item ghost code: is regular C-code, only visible from the
    specification, that is only allowed to modify ghost variables. See
    section~\ref{sec:ghost}. Ghost braces for enclosing blocks.

  \end{itemize}

\item Attribute annotations: \experimental. See
  Section~\ref{sec:attribute_annot}.

\end{itemize}

\subsection{Parsing annotations in practice}

In JML, parsing is done by just ignoring \verb|//@|, \verb|/*@| and
\verb|*/| and the level of lexing. This technique can completely
modify the semantics of the C code, for example: \input{annot1.pp}

In our language this is forbidden. Technically, the current
implementation of Frama-C isolates the comments in a first step of
syntax analysis, and then parses a second time. Nevertheless, the
grammar and the corresponding parser must be carefully designed to
avoid interaction of annotations with the code. For example, in such a
code: 
\input{annot2.pp} 
the statement \verb|c=1| must be understood as the \texttt{then}
branch of the \texttt{if}. This is ensured by the grammar below,
saying that \verb|assert| annotations are not statement themselves,
but attached to the statement that follows, like C labels.

\subsection{About preprocessing}

This document considers C source \emph{after} preprocessing. Tools
must decide what to do for preprocessing phase: what to do with
annotations, should macro substitution be performed or not, etc.

\subsection{About keywords}

Additional keywords of the specification language start with a
backslash, if they are used in position of a term or a predicate
(which are defined in the following).
Otherwise they do not start with a backslash, like \ensures{}.


\section{Notations for grammars}

In this document, grammar rules are given in BNF form. In grammar
rules, we use extra notations $e^*$ to denote repetition of zero, one
or more occurrences of $e$, $e^+$ for repetition of one or more
occurrences of $e$, $e^?$ for zero or one occurrence of $e$.



%%% Local Variables:
%%% mode: latex
%%% TeX-PDF-mode: t
%%% TeX-master: "main"
%%% End:


\chapter{Specification language}

\section{Function behavior}
\label{sec:fn-behavior}
Grammar:

\begin{syntax}
  fun-contract ::= simple-behavior named-behavior * decreases-clause ? 
  \
  simple-behavior ::= (requires-clause | assigns-clause |
  ensures-clause ) * 
  \
  named-behavior ::= "behavior" ident ":" behavior-body 
  \
  behavior-body ::= (assumes-clause | requires-clause | assigns-clause |
              ensures-clause ) * 
              \
  assumes-clause ::= "assumes" predicate ";"
  \
  requires-clause ::= "requires" predicate ";"
  \
  assigns-clause ::= "assigns" (location ("," location) * |
  "\nothing") ";"
  \
  ensures-clause ::= "ensures" predicate ";"
  \
  decreases-clause ::= "decreases" term ("for" ident)? ";"
\end{syntax}


\remark{Patrick}{pourquoi seuls les ``behavior'' peuvent avoir un
  nom~? Ne peut-t'on pas avoir le ``ident:'' optionnel pour chacune
  des clauses~?}

\remark{Virgile}{Changement du mot-cl� des variants pour �viter un
  conflit avec les variants de boucle. Le choix du mot-cl� exact reste
  � d�battre}

semantics:

\begin{verbatim}
  requires P_1
  requires P_2
  behavior x_1: assumes A_1 ensures E_1
  behavior x_2: assumes A_2 ensures E_2
\end{verbatim}


\begin{verbatim}
 pre-condition : P_1 and P_2
 post-condition: (\old(A_1) implies E_1)
             and (\old(A_2) implies E_2)
\end{verbatim}

\remark{Yannick}{JML propose deux modes de sp�cification des fonctions~:
 soit � base de ``requires'' et ``ensures'', soit � base de
 ``behavior''. JML n'a pas de ``assumes'' dans les
  ``behaviors'', mais des ``requires''. Dans le cas o� les deux modes
  peuvent se combiner, la s�mantique est la suivante~:}

\begin{verbatim}
  requires P_1
  requires P_2
  ensures  Q_1
  ensures  Q_2
  behavior x_1: requires R_1 ensures E_1
  behavior x_2: requires R_2 ensures E_2
\end{verbatim}


\begin{verbatim}
 pre-condition : P_1 and P_2
             and (R_1 or R_2)
 post-condition: Q_1 and Q_2
             and (\old(R_1) implies E_1)
             and (\old(R_2) implies E_2)
\end{verbatim}

\remark{Patrick}{les ``behavior'' de JML pr�sente l'avantage de
  pouvoir sp�cifier une fonction par cas (non exclusifs)
  et de v�rifier que les cas d'appel sont sp�cifi�s (sinon on ne peut
  v�rifier la pr�-condition).
  Il semble important � Airbus de pouvoir s'assurer qu'ils ont
  �nonc� l'ensemble des cas.}

\remark{Yannick}{avoir ``requires'' en plus des ``assumes'' dans les
  ``behaviors'' semble utile :
la s�mantique consiste � rajouter $(A_i \ra R_i)$ en conjonction de la
pr�condition globale}

\begin{verbatim}
  requires P_1
  requires P_2
  ensures  Q_1
  ensures  Q_2
  behavior x_1: requires R_1 assumes A_1 ensures E_1
  behavior x_2: requires R_2 assumes A_2 ensures E_2
\end{verbatim}


\begin{verbatim}
 pre-condition : P_1 and P_2
             and (A_1 implies R_1)
             and (A_2 implies R_2)
 post-condition: Q_1 and Q_2
             and (\old(A_1) implies E_1)
             and (\old(A_2) implies E_2)
\end{verbatim}

\subsection{Memory locations}
\label{sec:locations}

There are several places where one needs to describe a set of memory locations: \assigns{} clauses, \reads{} clauses.

We give now the grammar for denoting such a set of memory locations

\begin{syntax}
  tset ::= "\empty" ; empty set
       | tset "->" id ;
       | tset "." id ;
       | "*" tset ;
       | "&" tset ;
       | tset "[" tset "]" ;
       | [ term ".." term ] ; range
       | "\union" "(" tset ("," tset)* ")" ; union of locations
       | "\inter" "(" tset ("," tset)* ")" ; intersection
       | tset "+" tset ;
       | "(" tset ")" ;
       | [ "{" tset "|" binder ";" guards "}" ]; set comprehension
       | { "{" term "}" } ; explicit singleton
       | term ; implicit singleton
       \
  pred ::= {"\subset" "(" tset "," tset ")"} ; set inclusion
\end{syntax}


The semantics is given as follows, where $s$ denotes any tset:
\begin{itemize}
\item a simple term denotes a singleton
\item s->id denotes the set of x->id for each x in s
\item s.id denotes the set of x.id for each x in s
\item *s denotes the set of *x for each x in s
\item $s_1[s_2]$ denotes the set of $x_1[x_2]$ for each $x_1 \in s_1$ and $x_2 \in s_2$
\item t1 .. t2 denotes the set of integers between t1 and t2, included.
\item $s_1,s_2$ denotes the union of $s_1$ and $s_2$
\item $s_1+s_2$ denotes the set of $x_1+x_2$ for each $x_1 in s_1$ and $x_2 in s_2$
\item (s) denotes the same set as s
\item $\{ s \mid b ; P \}$ denotes set comprehension: set of term denoted by s foreach values of binders satisfying predicvate P. Binders b are bound in s and P.
\end{itemize}

A \emph{location} is any set of terms denoting a set of lvalues.
Only locations are valid as argument to \assigns{} clauses

Examples:
\begin{c}
assigns  (forall struct list *p ; reachable(root,p)) -> hd
\end{c}


\remark{Patrick}{ne faut-t'il pas �tendre les clauses ``assigns'' aux
  clauses ``from'' de CAVEAT prenant en comptes les locations lues, et les
  expressions fonctionnelles~?}

\subsection{Typing rules}

Two judgements:
\begin{itemize}
\item $\Gamma,\Lambda \vdash e : loc \tau$ means e is a set of location of type
  $tau$
\item $\Gamma,\Lambda \vdash e : tset \tau$ means e is a set of terms of type
  $tau$
\end{itemize}
$\Gamma$ is the C environment and $\Lambda$ is the logic environment.

Rules:
\[
\frac{\vdash e:loc \tau}{\vdash e: tset \tau}
\]
\[
\frac{\tau x \in \Gamma}{\vdash x: loc \tau}
\]
\[
\frac{e:tset \tau*}{\vdash *e: loc \tau}
\]
\[
\frac{e_1:tset \tau \quad e_2:tset \tau}{\vdash e_1,e_2: \tau}
\]
\[
\frac{e: tset struct S* \quad e_2:tset \tau}{\vdash e->f : loc \tau}
\]
\[
\frac{b\cup \Lambda \vdash e: tset \tau}{\vdash \{ e \mid b ; P \} : tset \tau }
\]
idem for $loc \tau$

Notes:
\begin{itemize}
\item Quantification can be be made over any type (both C and
  logic types).
\item Quantification over pointers, structures, union, etc.
  are possible too. TODO: the meaning must be carefully defined.
\end{itemize}

\section{Statement annotation}

  \begin{itemize}
  \item \assert, \assume, label
    \remark{claude}{permettre de nommer les asserts}
  \item loop invariants (+loop assigns ?)
  \item block behaviors (normal termination only, no goto going
    inside, no goto going outside)
  \end{itemize}

\subsection{Loop annotations}

semantics of loop invariants: in particular for {\tt for} loops

\begin{syntax}
  invariant-clause ::= "loop invariant" predicate \\
\end{syntax}

loop assigns: semantics?


\section{Expressions}
\label{sec:expressions}

logical expressions = pure C expressions, without function
  calls. always defined (2-valued logic). With additional constructs:
  \begin{itemize}
  \item values in preceedings states: \old(e) and \at(e,label).

    construct \old{} is only valid in ensures clauses of function
    specification. For loop invariants, one need to put some label and
    use the \at{} construct.
  \item returned value: \result

    valid only in ensures clauses of function
    specification.

  \item quantifications:  $\Forall \tau x; e$ and $\Exists \tau x; e$
  \item \verb|==>|, \verb|<==>|,

   \remark{Clause}{Open question:}
   \verb|=>| and \verb|<=>| instead of \verb|==>|, \verb|<==>| ?

   \remark{Patrick}{
   en C il existe le ou exclusif bit � bit (not� \texttt{\^}), mais pas
   d'implication bit � bit, ni d'�quivalence.
   Cot� contr�le (que l'on assimile facilement � la logique) il
   n'existe pas en C d'�quivalence (car c'est l'�galit� qui joue ce
   r�le), ni de ou exclusif. Ne d�sire-t'on pas avoir
   quelque chose du genre :}
   \begin{tabular}{cc}
    bit � bit & logique \\
    \verb|&| & \verb|&&| \\
    \verb|=>| & \verb|==>| \\
    \verb|<=>| & \verb|<==>| \\
    \verb|^| & \verb|^^|
    \end{tabular}

  \item $\Let \tau x = e \In e$ and $\If e \Then e \Else e$

  \item nommage de formules, de termes

  \end{itemize}

on veut la logique equationnelle classique, cad que l'axiome
\[
\forall x, x=x
\]
est valide. Donc on ne peut pas introduire de construction
non-d�terministe comme  $(\texttt{any} x \mid P)$


Types de la logique (see Section~\ref{sec:logicspec}:
\begin{itemize}
\item types mathematiques: real, integer, boolean
\item types du C
\item types logiques introduits par l'utilisateur
\end{itemize}


\subsection{Grammar of logical expressions}

\begin{syntax}
  bin-op ::= "+" | "-" | "*" | "/" | "%" ;
       | "==" | "!=" | "<=" | ">=" | ">" | "<" ;
       | "&&" | "||" |   ; boolean operations
       | "&" | "|" | {"--->"} | {"<--->"} | "^"; bitwise operations
       \
  unary-op ::= "+" | "-" ; unary plus and minus
       | "!" ; boolean negation
       | "~" ;  bitwise complementation
       | "*" ; pointer dereferencing
       | "&" ; address-of operator
       \
  term ::= "\true" | "\false" ;
       | integer ; integer constants
       | real ; real constants
       | id ; variables
       | unary-op term ;
       | term bin-op term ;
       | term "[" term "]" ; array access
       | "{" term "for" "[" term "]" "=" term "}" ; array functional modifier
       | term "." id  ; structure field access
       | "{" term "for" id "=" term "}" ; structure field functional modifier
       | term "->" id ;
       | "(" type-expr ")" term  ; cast
       | id "(" term ("," term)* ")" ; function application
       | "(" term ")" ; parentheses
       | term "?" term ":" term ;
       | {"\let" id "=" term ";" term} ; local binding
       | "sizeof" "(" term ")" ;
       | "sizeof" "(" C-type-expr ")" ;
       | id ":" term ; syntactic naming
       \
  rel-op ::= "==" | "!=" | "<=" | ">=" | ">" | "<"
       \
  pred ::= "\true" | "\false" ;
       | term (rel-op term)+ ; comparisons (see remark)
       | id "(" term ("," term)* ")" ; predicate application
       | "(" pred ")" ; parentheses
       | pred "&&" pred ; conjunction
       | pred "||" pred ; disjunction
       | pred "==>" pred ; implication
       | pred "<==>" pred ; equivalence
       | "!" pred ; negation
       | pred "^^" pred ; exclusive or
       | term "?" pred ":" pred ;
       | { pred "?" pred ":" pred };
       | { "\let" id "=" term ";" pred }; local binding
       | "\forall" binders ";" pred ; universal quantification
       | "\exists" binders ";" pred ; existential quantification
       | id ":" pred ; syntactic naming
\end{syntax}

%%% Local Variables:
%%% mode: latex
%%% TeX-master: "main"
%%% End:


\section{Pointers and physical adressing}

\subsection{Memory blocks and pointer dereferencing}

\begin{itemize}
\item \baseaddr{} base address of an allocated pointer
\[
\baseaddr{} : \alpha {\tt *} \ra {\tt char *}
\]
\item \blocklength{} length of the allocated block of a pointer
\[
\blocklength{} : \alpha {\tt *} \ra {\tt size\_t}
\]

\end{itemize}

Shortcuts:
\begin{itemize}
\item offset(p) returns the offset between p ans its base address

  \begin{eqnarray*}
    offset &:& \alpha {\tt *} \ra {\tt size\_t}  \\
    offset(p) &=& (char*)p - \baseaddr(p)
  \end{eqnarray*}

\item valid(p) tells whether dereferencing p is safe

  \begin{eqnarray*}
    valid : \alpha {\tt *} \ra {\tt boolean} \\
    valid(p) = offset(p) \geq 0 \land offset(p) + sizeof(*p) \leq \blocklength(p)
  \end{eqnarray*}

\item \comparable{} (checks whether two pointers are comparable as defined
  in the ANSI standard: TODO by benjamin)
\[
\comparable{} : \alpha {\tt *} \ra \beta {\tt *} \ra {\tt boolean}
\]
\end{itemize}


\subsection{Separation}

pointer separation :
\[
p \not\equiv q := \baseaddr(p) \neq \baseaddr(q) \lor |(char*)p - (char*)q| \geq \max(\sizeof(p),\sizeof(q))
\]

$\separated(loc1,..,loc_n)$ : means that for if $i\neq j$, if $x\in loc_i$ and $y \in loc_j$ then $\&x \not\equiv \& y$

where each $loc_i$ is a set of memory location as defined in Section~\ref{sec:locations}.

\subsection{Allocation and deallocation}

\experimental

\begin{itemize}
\item built-in predicate \fresh, specifying in a post-condition that a
  pointer was not allocated in the pre-state.
 
\item built-in predicate \freed, specifying in a post-condition that a
  pointer was allocated in the pre-state but not anymore.
\end{itemize}

\section{Termination}

Property of termination concerns both loops and recursive function calls. 
For that purpose, loops can be annotated with \emph{loop variants}\index{loop variant}, and functions can be annotated with such variants too.

\subsection{integer measures}

For loops:
\begin{syntax}
variant_clause ::= "loop variant" e 
\end{syntax}

For functions:
\begin{syntax}
decreases_clause ::= "decreases" e 
\end{syntax}
where $e$ has type integer

\subsection{general measures}

More general case of measures on other types: use the keyword for:

\begin{syntax}
variant_clause ::= "loop variant" e "for" id
decreases_clause ::= "decreases" e "for" id
\end{syntax}

where $e$ has type $\tau$ and $id$ is a logic predicate of type $\tau,\tau \ra
prop$ (see Section~\ref{sec:logicspec}


\section{Logic specifications}
\label{sec:logicspec}

\begin{itemize}
\item logic specifications: introduction of new logic types,
  constants, functions, predicates and axioms.

logic types can be polymorphic: for example
\begin{c}
type 'a list
\end{c}

\begin{syntax}
  C-external-declaration ::= "/*@" logic-def+ "*/"[These may appear as global \ifCPP{or class} declarations] ;
  \
  logic-def ::= logic-const-def ;
          | logic-function-def ;
          | logic-predicate-def ;
          | lemma-def ;
          | data-invariant-def;
  \
  type-var ::= id
  \
  type-expr ::= type-var ; type variable
  | id;
    "<" type-expr;
    (, type-expr)* ">" ; polymorphic type
  \
  type-var-binders ::= "<" type-var;
                       (, type-var)* ">"
  \
  poly-id ::= ident type-var-binders ; polymorphic object identifier
  \
  logic-const-def ::= "logic" type-expr;
    poly-id "=" term ";"
  \
  logic-function-def ::= "logic" type-expr;
  poly-id parameters "=";
  term ";"
  \
  logic-predicate-def ::=
  "predicate";
  poly-id parameters? "=";
  pred ";"
  \
  parameters ::= "(" parameter;
                 (, parameter)* ")"
  \
  parameter ::= type-expr id
  \
  lemma-def ::= "lemma" poly-id ":";
                   pred ";"
\end{syntax}


\item recursive definitions: are allowed in logic function and predicate definitions. Example:  
  \begin{c}
    logic int max\_index(int t[],int n) {
      (n==0) ? 0 :
      (t[n-1]==0) ? n : max\_index(t, n-1)
    }
  \end{c}

mutual recursion

\item predefined logic specifications can be provided as libraries (see section~\ref{sec:speclibraries}, and imported using
\begin{c}
//@ import <specfile>
\end{c}

\item higher-order functions: $\lambda$, \Sum, \Max, \numof
\item {\tt enum}, recursive types


\item hybrid functions and predicates: take both C types and logic
  types as arguments.
\item model variables
\end{itemize}


\section{Abnormal termination}

\begin{itemize}
\item C function \verb|exit(n)|. special behavior notation
  \begin{verbatim}
    exit_behavior
       assumes
       assigns
       ensures ...
  \end{verbatim}
    where in ensures clause, \result is bound to n
\end{itemize}


\section{Dependencies information}

\begin{verbatim}
assigns Locs from Locs
\end{verbatim}

\section{Functional expressions}

trouver une syntaxe pour introduire les noms des fonctions implicites
derriere la construction
\begin{verbatim}
assigns Locs from Locs
\end{verbatim}

\section{Data invariants}
\label{sec:invariants}

\begin{itemize}
\item Global invariants: invariants on global variables
\item Type invariants: invariants on struct, union, typedef
\end{itemize}

Examples:
\begin{c}
int a;
//@ global invariant a_is_positive: a > 0

typedef int T;
//@ type invariant T_is_positive(T x) { x > 0 } 

struct S {
  int f;
}
//@ type invariant S_f_is_positive(struct S s) { s.f > 0 } 
\end{c}


\subsection{Semantics}

an invariant is true when ...\cite{barnett04jot}


\section{Ghost variables and statements}
\label{sec:ghost}

\remark{Patrick}{les ``ghost statements'' correspondent � des
instructions d'observations des variables du C. Ces instructions
ne peuvent pas modifier les variables du C, mais que les ``ghost
variables''. Cela permet d'�crire plus facilement l'observateur
ad�quat � la preuve des propri�t�s puisque les propri�t�s peuvent
porter � la fois sur les variables du C et les  ``ghost variables''.
}


\begin{syntax}

  ghost-type-specifier ::= C-type-specifier ;
  | {logic-type} \
  logic-def ::= "ghost" ghost-declaration \
  direct-declarator ::= C-direct-declarator ;
    | direct-declarator ;
    "(" C-parameter-type-list? ")";
        "/*@" "ghost";
          "(" ghost-parameter-type-list")";
          "*/"; ghost args
        \
  postfix-expression ::= C-postfix-expression ;
    | postfix-expression ;
     "(" C-argument-expression-list? ")";
     "/*@" "ghost" ;
       "(" ghost-argument-expression-list ")";
       "*/" ; call
              ; with ghosts
    \
  statement ::= C-statement ;
             | statements-ghost \
  statements-ghost ::= "/*@" "ghost";
                       ghost-statement+ "*/" \
  ghost-selection-statement ::= C-selection-statement ;
    | "if" "(" C-expression ")";
       statement;
      {"/*@" "ghost" "else"};
      {  ghost-statement+ };
      {  "*/"} \

  struct-declaration ::= C-struct-declaration ;
  | {"/*@" "ghost" };
    { struct-declaration "*/"} ; ghost field

\end{syntax}

%%% Local Variables:
%%% mode: latex
%%% TeX-master: "main"
%%% End:



\section{Module constructions}

how to encapsulate several functions...


\section{Arithmetic, Overflow}


quantification can be either on mathematical \verb|integer| or bounded
types \verb|short|, \verb|char|, etc.

we need macros \verb|\max_range|, \verb|min_range| taking a C integer
type as argument, e.g. \verb|\max_range(unsigned char) = 255|\\

\remark{Patrick}{Ces marcros, types et variables sont en principe
d�finies dans des \texttt{.h} que la norme sp�cifie en grande partie
(le nom l'est, le type peut y �tre contraint).
Il faudrait autant que possible se raprocher de ces noms.}


\chapter{Librairies}

\section{Jessie library: logical adressing of memory blocks}

Definition: a \emph{valid} pointer is a pointer such that *p is properly
allocated.

\offsetmin, \offsetmax : $\alpha * \ra \N$

$\offsetmin(p)$: the minimum integer $i$ such that $(p+i)$ is a valid
pointer

$\offsetmax(p)$: the maximum integer $i$ such that $(p+i)$ is a valid
pointer

properties:

\[
\offsetmin(p+i) = \offsetmin(p)-i
\offsetmax(p+i) = \offsetmax(p)-i
\]

syntactic sugar:
\[
\validrange(p,i,j) := \offsetmin(p) <= i \land \offsetmax(p) >= j
\valid(p) := \validrange(p,0,0)
\]

\remark{Benjamin}{propose plutot les noms ``indexmin'' et ``indexmax'' a la place de offsetmin et offsetmax}

\subsection{Strings}

\strlen

Peut disparaitre

\section{Memory leaks}

\experimental

Verification of absence of memory leak is outside the scope of the
specification language. On the other hand, various models could be set
up, using for example ghost variables.


\chapter{Quick reference card}

TODO by Benjamin

\section{Libraries of logic specifications}

\subsection{Library list}

\begin{verbatim}
type 'a list = ...
\end{verbatim}

\chapter{Comparison with JML}

TODO by Yannick

\begin{verbatim}
  JML                  Frama-C

  loop_invariant       loop invariant 
  decreases            loop variant

\end{verbatim}

\bibliographystyle{plain}
\bibliography{./biblio}
%\input{biblio-demons}

\tableofcontents

\printindex
\end{document}

%%% Local Variables:
%%% mode: latex
%%% TeX-master: t
%%% End:
