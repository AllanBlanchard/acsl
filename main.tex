\documentclass{article}

\newcommand{\assert}{\texttt{assert}}
\newcommand{\assume}{\texttt{assume}}

\begin{document}

\emph{pure} : side-effect free

\emph{behavior}: pair pre- and post-condition, assigns

\section{Frama-C base level}

\begin{itemize}
\item logical expressions = pure C expressions. always defined
  (2-valued logic). + \old, \result, \forall, \exists, ==>, <==>,
  \old(e,label), let \tau x = e in e, if e then e else e, 
\item C function behaviors
\item statement annotations:
  \begin{itemize}
  \item \assert, \assume, label
  \item loop invariants (+loop assigns ?)
  \item block behaviors (normal termination only, no goto going
    inside, no goto going outside) 
  \end{itemize}
\end{itemize}

\begin{alltt}
  fun-spec ::= (requires-clause | fun-behavior ) * variant-clause ?

  requires-clause ::= "requires" predicate
  variant-clause ::= "variant" term ("for" ident)?

  fun-behavior ::= "behavior" ident ":" 
             (assumes-clause | assigns-clause | 
              ensures-clause | invariant-clause) *

  assumes-clause ::= "assumes" predicate
  assigns-clause ::= "assigns" (location (, location) * | "\nothing")
  ensures-clause ::= "ensures" predicate
  invariant-clause ::= "invariant" predicate

  location ::= (see below)
\end{alltt}

semantics:
 
  requires R_j
  behavior x_i: assumes A_i ensures E_i invariant I_i


 pre-condition : R_1 and R_2 and ... 
 post-condition : (\old(A_1 and I_1) implies E_1 and I_1) and ...

  
  



Questions: "reads" clause ?

\subsection{syntax of assigns clauses}

\begin{alltt}
  location ::= 
       var | location->id | location.id | *location 
       | location[term ? .. term?] | location[term] 

       |  (\all \tau) | (\all \tau x ; predicate) 
       |  location + [ term? .. term? ] | location + term         
\end{alltt}

semantics:
\begin{itemize}
\item a "location" denotes a set of terms. 
\item only location denoting a set of lvalues are valid as argument to
  "assigns" clauses : constructs in the two first lines above denotes
  lvalues, but not the two last lines
\end{itemize}

example:

\begin{alltt}
  assigns  (\all struct list *p ; reachable(root,p)) -> hd
\end{alltt}

\section{Constructions for valid pointer dereferencing, separation}

\subsection{Memory blocks: physical adressing}

\verb|\base_addr|
\verb|\blocklength|


\subsection{Memory blocks: logical adressing}

a \emph{valid} pointer is a pointer such that *p is properly
allocated.

\offsetmin, \offsetmax : \alpha * \ra \N

\verb|\offset_min(p)| : the minimum integer i such that (p+i) is a valid
pointer

\verb|\offset_max(p)|: the maximum integer i such that (p+i) is a valid
pointer

properties: 

\[
\offsetmin(p+i) = \offsetmin(p)-i
\offsetmax(p+i) = \offsetmax(p)-i
\]

syntactic sugar:
\[
\validrange(p,i,j) := \offsetmin(p) <= i \land \offsetmax(p) >= j
\valid(p) := \validrange(p,0,0)
\]

\subsection{Strings}

\strlen

\subsection{Separation}

pointer separation :
\[
p \not\equiv q := \baseaddr |(char*)p - (char*)q| \geq \max(\sizeof(p),\sizeof(q))
\]

\separated(p1,..,p_n) : means that for if i<>j, p_i \not\equiv p_j

\full_separated

\bound_separated(p,n,q,m)

\section{Termination}

termination: loop variants, function variants for recursive
  functions, for integer measures.

\section{Logic specifications}

\begin{itemize}
\item logic specifications: introduction of new logic types,
  constants, functions, predicates and axioms
\item polymorphism
\item higher-order functions: \lambda, \sum, \max, \num_of
\item types enumer�s
\item types recursifs, definitions recursives (?)
\item hybrid functions and predicates: take both C types and logic
  types as arguments.
\item model variables
\end{itemize}

\section{Allocation and deallocation}

\fresh, \freed

\section{Abnormal termination}

\begin{itemize}
\item C function \verb|exit(n)|. special behavior notation
  \begin{alltt}
    exit_behavior
       assumes
       assigns
       ensures ... 
  \end{alltt}
    where in ensures clause, \result is bound to n
\end{itemize}


\section{Dependencies information}

assigns Locs from Locs

\section{Data invariants}


\begin{itemize}
\item Global invariants: invariants on global variables
\item Type invariants: invariants on struct, union, typedef 
\end{itemize}

semantics

an invariant is true when ... \cite{boogie}

\section{Ghost variables and statements}










\end{document}

%%% Local Variables: 
%%% mode: latex
%%% TeX-master: t
%%% End: 
