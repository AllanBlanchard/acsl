\section{Functional programming [C/C++]}
\label{sec:functional}

\experimental

\subsection{C function pointers}

In C (and C++) function pointers can be formal arguments to other functions and can be invoked without knowing which functional literal the pointer refers to. 
In order to reason about such constructs, \NAME needs means to describe and reason about the properties of function pointers. The feastures and design presented here are
described more fully in \S\ref{sec:functional}.

The essential properties of function pointers are its specifications. For a function pointer \lstinline|&f| for a known function \lstinline|f|,  we could write \lstinline|\requires(&f)| to represent the predicate that is the precondition of \lstinline|f|. However this then requires the specification language to treat predicates as first class elements and its reasoning becomes higher-order. Instead we follow \cite{20.500.11850/153152} and write, for a function \lstinline|f| with arguments \lstinline|...args|, the expression \lstinline|\requires(&f, ...args)| to be equal to the  value of
the precondition of \lstinline|f| when evaluated at the arguments \lstinline|,...args|.

In particular, if \lstinline|f| has the specification
\begin{listing-nonumber}
requires R0;
ensures E0;
assigns L0;
behavior b1:
  assumes A1;
  requires R1;
  ensures E1;
  assidgns L1;
  allocates C1;
  frees F1;
behavior b2:
  assumes A2;
  ...
\end{listing-nonumber}
where each of the clauses is an expression over the formal arguments and perhaps other variables in scope, then

\begin{listing-nonumber}
\requires(f,...args) = R0 && (A1==>R1) && (A2==>R2) ...
\ensures(f,result,...args) = E0 && (A1 ==> E1) && (A2 ==> E2) ...
\assigns(f,...args) = E0 && (A1 ==> L1) && (A2 ==> L2) ...
\end{listing-nonumber}
In the case of \lstinline|\ensures|, the extra parameter, here labeled \lstinline|result| corresponds to the \lstinline|\result| identifier in the \lstinline|Ei| expressions.
There is a backslash-keyword for each kind of function contract clause (as shown in Fig. \ref{fig:gram:functional}).

As an example, if we have a function such as 
\begin{listing-nonumber}
/*@ requires i < 100;
       assigns \empty;
       ensures \result == i+1;
       */
int inc(int i) { return i + 1; }
\end{listing-nonumber}
Then, for this function, for all \lstinline|int k|
\begin{listing-nonumber}
\requires(inc, k) == k < 100;
\assigns(inc, k) == \empty;
\ensures(inc, result, k) == (result == k+1);
\end{listing-nonumber}
The other functions with one formal int parameter and an int result would have analogous definitions for \lstinline|\requires|, etc.

An example function that takes such functions as arguments is this.

\lstinputlisting[linerange={7-9}]{func_args.cpp}

It could have this specification.

\lstinputlisting[linerange={1-6}]{func_args.cpp}

Then the effect of, say the call \lstinline|m(a,4,7,inc)|, can be determined by logical reasoning.

These specifications only apply for the case of the actual parameter being a function with no side-effects. More complex specifications (and perhaps additional specification language features) are still under experimentation.

\subsection{C++ member and virtual functions}

Where the function in question is a method of a class, there are two additional considerations: there is an implicit receiver and the method may be virtual.

The implicit receiver is accommodated by inserting an additional first argument. For example if we had a method
\begin{listing-nonumber}
class C {
  int m(int k) { ... }
};
\end{listing-nonumber}
then the  \lstinline|\requires| predicate for the member method looks like \lstinline|\requires(pointer_to_C, &C::m, k) == ... |.

Now suppose \lstinline|C::m| is virtual. The example above gives the precondition for method \lstinline|C::m|, even if that method is virtual. However, there is also a need at times to express the dynamic precondition. In that case, the syntax is to place a receiver prior to the keyword, with either a dot or an arrow as appropriate (again, cf. Fig. \ref{fig:gram:functional}).


\subsection{Specifying lambda expressions}

The difficulty with specifying lambda expressions is that the lambda is an expression, not a statement, and thus there is not necessarily a convenient location to write a specification.
C++ lambda expressions however do include a block statement; they have syntax of the form\\
\centerline{ [ \textit{captures} ] ( \textit{parameters} ) -> \textit{return-type}  \{ \textit{lambda statements} \} }
The specification of the lambda function can be placed immediately before the opening brace, as in \\
\centerline{ [ \textit{captures} ] ( \textit{parameters} ) -> \textit{return-type} /*@ function-contract */ \{ \textit{lambda statements} \} }

\begin{figure}
\begin{cadre}
\begin{syntax}
spec-keyword ::= "\requires" | "\ensures" | "\assigns"  ;
    | "\allocates" | "\frees" | "\throws" | "\exits" 
\
term ::= spec-keyword "(" term ("," term)* ")" 
\
term ::= term ( "->" | "." ) spec-keyword "(" term ("," term)* ")" 
\end{syntax}

\end{cadre}
\caption{Grammar elements to support specification of functors}
\label{fig:gram:functional}
\end{figure}
