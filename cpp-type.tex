\section{Types}
\label{sec:type}
\experimental

\lang{} includes the \lstinline|typeid| operator, which produces values of
type \lstinline|type_info|, which in turn uniquely identify types within
a \lang{} program. These \lstinline|type_info| objects are immutable and are suitable for comparing types in logic expressions as well as in \lang.
The type information is the full type information, and not subject to erasure,
as for example in Java.

\lang{} does provide static, template-based operations that report variuos traits of types. But these are static and do not enable reasoning about 
dynamic types represented by \lstinline|type_info| values.
\lang{} does not define any operations on \lstinline|type_info| values other than equality and conversion to a (arbitrary) name. As an exploratory experiment
\NAME{} defines some such operators, corresponding to those in \lstinline|type_traits|:
\begin{itemize}
\item  \lstinline|\is_integral|
\item \lstinline|\is_primitive|
\item \lstinline|\is_object|
\item \lstinline|\is_subclassof|
\item \lstinline|\is_array|
\item \lstinline|\is_enum|
\end{itemize}
