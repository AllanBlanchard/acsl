\section{Class contracts [C++]}
\label{sec:class-contracts}

C++ allows the declaration of functions and data fields within
the scope of the declaration of an aggregate type (class, struct or union). Similarly \NAME specification annotations
may be present at class (or struct or union) scope.

\subsection{Global declarations}

As described in Section \ref{sec:logicspec}, logical specifications (predicates, functions, lemmas, and constants) may be defined at global scope. They may also be defined in class (that is, aggregate) scope. The following considerations apply:
\begin{itemize}
	\item Just as for \lang identifiers, any names 
	declared are part of the aggregate scope. The name may
	be referred to outside of the aggregate using \lang's 
	qualification syntax; for example, \lstinline|A::m| names a 
	logic predicate declared as \lstinline|m| in class
	or namespace \lstinline|A|.
	\item As for any other specification constructs, those defined in aggregate scope are only visible in specification annotations. However, logic declarations are always public.
	\item Within aggregate scope, a predicate or function definition may be
	declared \lstinline|static|, with the same meaning as 
	\lstinline|static| has for \lang function 
	declarations: non-static logic predicates and functions
	may refer to \lstinline|this| and are invoked with the
	\lang arrow or dot syntax.

	\item Logic lemmas, logic type definitions, axiomatic definitions, and logic constant definitions are always implicitly static. The effect of declaring them within an aggregate is just one of name scope.
\end{itemize}

\TODO{Should give some significant examples}

\subsection{Function contracts within class scope}

Function declarations or definitions that belong to an aggregate (i.e., member functions) may have function
contracts just like functions declared or defined at global scope. If the function is not declared \lstinline|static|, then
the keyword \lstinline|this| may be used within the function
contract and within any statement contract within the 
function. The keyword \lstinline|this| refers to the current
object, just as \lstinline|this| does within \lang code.
The type of \lstinline|this| in contracts is the same as the type of \lstinline|this| in C++ code: \lstinline|T*| or \lstinline|const T*|, where \lstinline|T| is the type of the enclosing aggregate.

\TODO{TODO: Is this allowed to be used in constructor preconditions?}

\subsection{\textbf{this} and \textbf{\textbackslash this}}


\begin{figure}
\begin{cadre}
\begin{syntax}
term ::= "this" ;
  | "\\this" ;
  | term ( "." | "->" ) id "(" ( term ( "," term )* )? ")" ;
\
ident ::= ( qualifier "::" )* qualifier ;
\
poly-id ::= ( qualifier "::" )+ poly-id ;
\
qualifier ::= id ( "<" type-expr ( "," type-expr )* ">" )? ;
\
literal ::= "nullptr" | "true" | "false" ;
\end{syntax}

\end{cadre}
\caption{ACSL++ grammar enhancements from C++}
\label{fig:gram:this}
\end{figure}

Within C++ class methods, the identifier \lstinline|this| represents a
pointer to the object instance on which a method is acting. \NAME
allows both \lstinline|this| and this new construct \lstinline|\this| 
to be used as follows.
\begin{itemize}
\item \lstinline|this| designates, as in C++, a pointer to the object
instance a method is acting on. It has type either \lstinline|T*|
or \lstinline|const T*| per C++ rules (where \lstinline|T| is the type
of the class owning the method). \lstinline|this| may be used
within function contracts or statement contracts of non-static methods.
\lstinline|this| is a C++ entity permitted to be used in a logic context
and subject to interpretation according to the appropriate state labels.

\item \lstinline|\this| designates a \textit{reference} to the object
instance on which a method is acting. It has type either \lstinline|T&| or
\lstinline|const T&|. 
\lstinline|\this| may be used in logic definitions
such as predicates, functions, constants, and class invariants. Axioms, lemmas, type definitions, global type invariants and data innvariants
do not have an implicit receiver.

\end{itemize}

In C++, the fields and member functions of class objects are designated using 
either dot or arrow notation. Within \NAME annotations, \lang member functions are not
permitted. However, \NAME allows logic functions, predicates, and class invariants to be defined
within class scope and then called using conventional \lang dot and arrow notation.
Axioms, lemmas, type definitions, data and type invariants are not considered class members with implicit receivers,
but may be designated by appropriate \lstinline|::| qualifiers.

\subsection{Scoping}

\lang permits user control of identifier scopes and name conflicts by two mechanisms: class/structs and namespaces. 
Names declared within a namespace or class/struct are available outside the scope only when qualified by the
\lstinline|::| double-colon syntax. This standard \lang syntax is available within \NAME annotations as well.
The following code example shows some uses. Such syntax is typically needed only for \textit{static} entities.
Entities that are members of their enclosing class have a receiver and are designated with dot or arrow syntax;
the names are then found in the (static) class of the receiver term. In addition if a containing class is a template
instance, the qualifier will have type arguments. The corresponding grammar enhancements are shown in Fig. \ref{fig:gram:this}.
\begin{lstlisting}
namespace A {
   int f; // A::f
   class B {
      static int f; // A::B::f
      
      //@ predicate pos1() = f > 0;  // A::B::f
      //@ predicate pos2() = A::B::f > 0;  // A::B::f
      //@ predicate pos3() = A::f > 0;  // A::f
      //@ predicate pos4() = B::f > 0;  // A::B::f   
   }
   template <typename T> class C {
      class D {}
   }
}

predicate pos5() = f > 0;  // not legal
predicate pos6() = A::f > 0;  // A::f
predicate pos7() = A::B::f > 0;  // A::B::f

A::B::C<int>::D x;
\end{lstlisting}|

\NAME axiomatics do not create a nested scope.

\TODO{TODO: Order of name lookup}

\TODO{TODO: C++ using statement}

\subsection{Ghost functions}
Aggregates may also contain declarations of ghost functions, as described in section \S\ref{sec:ghost} \TODO{Complete this subsection}


\subsection{Inheritance of function contracts}

A key aspect of object-oriented programming is 
that one can derive classes from parent classes and in 
the process inherit behaviors of the parent class, 
while specializing the implementation to something appropriate to the derived class. 
To use a standard example, 
one might define an abstract base class \lstinline|Shape| to describe arbitrary 2D geometric shapes, 
with an abstract method 
\lstinline|area| that returns the area of the shape. 
Derived classes \lstinline|Circle| and \lstinline|Square| would contain concrete data fields that defined the parameters of circle and square shapes respectively; 
the derived classes would each have its own implementation of the area() method appropriate to its kind of shape. 
However, the key idea is that methods of a \lstinline|Shape| object can be used without knowing which derived class the object is actually an instance of and
which derived class method implementation is actually invoked.

The usual design intent is to obey Liskov and Wing's principle of \textit{behavioral subtyping}\cite{Liskov:1994:BNS:197320.197383}: 
anything provable about a base type should be provable about a subtype.
\NAME imposes that requirement on derived classes by 
requiring that each method of a derived class obeys the contract of any methods of parent classes that the derived class method overrides. 
If behavior subtyping does not hold between a base and derived class, then the behavior of methods
invoked on a pointer or reference whose static type is the
parent class (but whose dynamic class is any derived class) may depend on the actual dynamic class, 
a complex situation for program understanding, debugging, or verification.

Consider a derived class method \lstinline|D::m| that overrides a parent class method \lstinline|P::m|.
Behavioral subtyping is obeyed if (a) the (composite) precondition of
\lstinline|P::m| implies the (composite) precondition of 
\lstinline|D::m| and (b) the (composite) postcondition of
\lstinline|D::m| implies the (composite) postcondition of 
\lstinline|P::m|. That is, a derived class method satisfies the precondition stated in the parent class, but may be more lenient; 
postconditions in derived classes may be more strict than the parent class postcondition, but will always 
satisfy the parent postcondition. 

This fairly straightforward requirement becomes complex in its interaction with function contracts that have multiple behaviors, 
as described in \S \TODO{Sec ref needed}.
\footnote{JML was designed explicitly around behavioral subtyping and the implications of behavioral subtyping are much simpler. 
	See the comparative discussion in Appendix \ref{sec:comp-jml}.}

\TODO{TODO: More + examples needed}

\subsection{Specifications of operator overloads}
Declarations of operator overloading as treated in \NAME just like other
methods. In particular they can have function contracts as in this example:
\begin{lstlisting}
class MyType {
private:
    int myint;
public:
    MyType(int i): myint(i) {}

    //@ ensures \result.myint == this->myint + other.myint;
    MyType operator+(MyType& const other) {
        return MyType(myint + other.myint);
    }
}

/*@
@    requires \true;
@    assigns \nothing;
@    ensures this->value == that->value;
@ C(const C& that);
@*/
}
\end{lstlisting}
\subsection{Specifications of special member functions}

\lang implicitly declares and provides a default implementation for some member functions under some circumstances. 
Thus they also have an implicit specification. However, the user may also wish to provide an explicit specification for such implicit functions, but there is no
explicit declaration in the \lang code to which to attach such a 
specification.

In such a case, a corresponding declaration may be written as a \NAME 
declaration, with an attached specification. 
In the following example, class \lstinline|C| has an implicitly defined
copy constructor. The \NAME comment gives an explicit specification.
No body is given for the declaration of the \lang copy constructor.
Such declarations are only valid if the rules of \lang state that such
a member function is implicitly defined.
\begin{lstlisting}
class C {
  public:
  //@ logic integer value;
  
  /*@
    @    requires \true;
    @    assigns \nothing;
    @    ensures this->value == that->value;
    @*/
  C(const C& that) = default;
}
\end{lstlisting}
This syntax leverages the C++11 \lstinline|default| keyword, which indicates that the built-in default implementation for the construcxtor (or other special member function) is to be used. Thus we can attach a specification to 

\emph{Grammar additions are needed}

The member functions for which implicit default functions are provided by the compiler are the following. In each case the default action (and corresponding implicit specification) is to apply the corresponding operation on each base class and data member of the object at hand.
\begin{itemize}
	\item default constructor \lstinline|C()|: 
	\item default copy constructor \lstinline|C(C&)| or \lstinline|C(const C&)|
	\item default move constructor \lstinline|C(C&&)| (since C++11)
	\item default copy assignment operator \lstinline|C::operator=(C&)| or \lstinline|C::operator=(const C&)|
	\item default move assignment operator \lstinline|C::operator=(C&&)| (since C++11)
	\item default destructor \lstinline|C::~C()|
\end{itemize}

\TODO{TODO: What actually are the default specs}

The \lstinline|operator new(...)| and 
\lstinline|operator delete(...)| functions have the appearance of being implicitly defined. However, they are defined in the standard library
(include file \lstinline|new|). Their specifications are given there as
part of the specifications of the standard library.

\subsection{Deleted and defaulted definitions}

 \lang functions can be defined to be either deleted or defaulted, as in this code:
 \begin{lstlisting}[deletekeywords={default}]
 C::C(const& C) = delete; 
 C::C(const& C) = default; 
 \end{lstlisting}
 A definition that is \lstinline|=delete| results in a program for which any use of the given method results in a compile-time error,
 as if the function is not at all defined. It can be used to preclude the
 generation of otherwise implicitly generated functions.
 
 A definition that is \lstinline|=default| results in the corresponding
 function defined to have the body it would have as an implicit default
 member, even if the \lang rules would otherwise preclude the 
 function from being implicitly generated. That is, it forces 
 generation of the member function where it would not otherwise be implicitly generated.
 
Both of these features are handled by \lang and do not require 
extra features in \NAME. Only, if a method is implicitly generated, whether by using \lstinline|=default| or not, an implicit specification must be correspondingly generated.
