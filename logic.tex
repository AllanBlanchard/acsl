\begin{syntax}
  logic-decl ::= logic-type-decl ;
          | logic-const-decl | logic-const-def ;
          | logic-predicate-decl | logic-predicate-def ;
          | logic-function-decl | logic-function-def ;
          | axiom-def
  \\
  logic-type-decl ::= "logic" "type" logic-type ( = logic-type-def)? ";" ; ``typedef'' au lieu de ``type'' ?
  \\
  logic-type ::= id | type-var id ;
                 | "(" type-var (, type-var)* ")" id
                 \\
  type-var ::= "'" id
  \\
  logic-type-def ::= TODO by Claude
  \\
  logic-predicate-decl ::= "predicate" id fun-parameters ("reads" locations)? ";"
  \\
  logic-predicate-def ::= "predicate" id fun-parameters "{" pred "}"
  \\
  logic-function-decl ::= "logic" type-expr id fun-parameters ("reads" locations)? ";"
  \\
  logic-function-def ::= "logic" type-expr id fun-parameters "{" pred "}"
  \\
  type-expr ::= logic-type-expr | C-type-expr
  \\
  logic-type-expr ::= built-in-logic-type | id | type-expr id ;
                 | "(" type-expr (, type-expr)* ")" id
                 | logic-type-expr ("*" logic-type-expr)+  ; product type
  \\
  built-in-logic-type ::= "boolean" | "integer" | "real"
  \\
  fun-parameters ::= "(" parameter (, parameter)* ")"
  \\
  parameter ::= type-expr | type-expr id | parameter "[]" | ... ; TODO: faire comme en C
  \\
  logic-const-decl ::= "logic" type-expr id
  \\
  logic-const-def ::= "logic" type-expr id "=" term
  \\
axiom-def ::= "axiom" id ":" pred ";"
\end{syntax}