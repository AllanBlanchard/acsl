\section{Enums [C++]}
\label{sec:enums}
\index{enum}

\TODO{Needs grammar additions}

\lang{} introduced enum types that are more strongly checked than in C. \lang{} allows the declaration of \textit{unscoped} and \textit{scoped} \lstinline|enum| types. In either case, the underlying type of the enum is some integral type. Accordingly, \lang{} still allows some conversions between enum values and integral values. Unscoped enumerations have an implicit conversion to an int, as in,

\begin{listing-nonumber}
enum color { red, orange, yellow}; // Unscoped type, 
                        // because there is no 'struct' or 'class' keyword
color c1 = red;
color c2 = 1; // Illegal in C++
color c3 = (color)10; // Cast is permitted
int j = red; // Allowed for unscoped enum types
\end{listing-nonumber}

However, scoped enumeration values are not implicitly cast to integral types:

\begin{listing-nonumber}
enum struct color { red, orange, yellow}; // Scoped type, 
                        // because of the 'struct' or 'class' keyword
color c1 = red;
color c2 = 1; // Illegal in C++
color c3 = (color)10; // Cast is permitted
int j = color::red; // Forbidden for scoped enum types
int k = (int)color::red; // Cast is permitted
\end{listing-nonumber}

Note that in C++ an integral value may be cast to an enum value even if the value does not correspond to one of the named
enumerator values, as long as it falls within the implementation range of the enumeration.

In some situations, however, the author would like an enum type to be strictly distinguished from integral types in one or both of two respects:
\begin{itemize}
\item To forbid any conversions to or from integral values, even with casts.
\item To forbid creating any enumerator values that are not named in the enumeration declaration.
\end{itemize}

\NAME{} allows two modifiers on an enum declaration: \lstinline|strong| and \lstinline|complete|. If these are present on any declaration or definition of an enum, they must be present on all declarations and definitions of the enumeration type.
\begin{itemize}
	\item \lstinline|complete| means that no integral values are permitted that do not correspond to a named enumerator. A declaration of \lstinline|complete| means that any conversion from an integral value to an enum value 
	creates a proof obligation that the integral value belongs to one of the declared enumerators.
	\item \lstinline|strong| means that the enum is strongly
	typed, that is, that no conversions to or from integral
	values are permitted. This check can be performed during
	type checking and does not create any proof obligations.
	A \lstinline|strong| enum forbids any conversion from an integral value
	so it is also necessarily \lstinline|complete|.
\end{itemize}
