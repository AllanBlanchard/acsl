\subsection{Class invariants [C++]}

\begin{figure}[t]
\begin{cadre}
\begin{syntax}
  data-inv-def ::= class-invariant ;
  \
  class-invariant ::= {inv-strength?} "class" "invariant" 
                      id "=" pred ";" 
\end{syntax}

\end{cadre}
\caption{The grammar for class invariants}
\label{fig:gram:classinvariant}
\end{figure}

In C++, classes and structs are often used to encapsulate a data structure, 
providing a uniform means of reading and modifying properties of the data
structure and hiding implementation details. Often the implementation
data structure is expected to satisfy some well-formedness properties, such as
that various fields have mutually-consistent values or have values within
some subrange of their declared type.

Such properties can be expressed using \textit{class invariants}. Class invariants
are a form of type invariant, as shown in Fig. \ref{fig:gram:classinvariant}.
Such declarations must appear within a class or struct declaration and 
apply to that particular type. In other respects they are similar to 
type invariants: they are implicitly assumed and asserted at various
program points.

Just like type invariants (Section \ref{sec:invariants}), class invariants can be \lstinline|weak| or \lstinline|strong|, with the same semantics as for type invariants.
