\chapter{Libraries}
\label{chap:lib}

Disclaimer: this chapter is unfinished, it is left here to give
an idea of what it will look like in the final document.

This chapter is devoted to libraries of specification, built upon the
ACSL specification language.  Section~\ref{sec:jessie} describes
additional predicates introduced by the Jessie plugin of Frama-C, to
propose a slightly higher level of annotation.

\section{Libraries of logic specifications}
\label{sec:speclibraries}

A standard library is provided, in the spirit of the List module of
Section~\ref{sec:specmodules}


\subsection{Real numbers}
\label{sec:libreal}

A library of general purpose functions and predicate over real
numbers, floats and doubles.

Includes

\begin{itemize}
\item abs, exp, power, log, sin, cos, atan, etc. over reals

\item isFinite predicate over floats and doubles (means not NaN nor infinity)

\item rounding reals to floats or doubles with specific rounding modes.

\end{itemize}

\subsection{Finite lists}

\begin{itemize}
\item pure functions nil, cons, append, fold, etc.
\item Path, Reachable, isFiniteList, isCyclic, etc. on C linked-lists.
\end{itemize}


\subsection{Sets and Maps}

Finite sets, finite maps, in ZB-style.

\oldremark{PC}{

  proposer des syntaxes concretes pour des types logiques standards:
  ensemble en particulier, style ZB

}

\section{Jessie library: logical addressing of memory blocks}
\label{sec:jessie}

%Jessie is a plugin of the Frama-C platform, which connects to the
%corresponding jessie tool of the Why platform~\cite{filliatre07cav}
%for deductive verification of behavioral properties of programs.

The Jessie library is a collection of logic specifications whose
semantics is well-defined only on source codes free from
architecture-dependent features. In particular it is currently
incompatible with pointer casts or unions (although there is ongoing
work to support some of them~\cite{moy07ccpp}). As a consequence, a valid
pointer of some type $\tau*$ necessarily points to a memory block which
contains values of type $\tau$.

\subsection{Abstract level of pointer validity}

In the particular setting described above, it is possible to
introduce the following logic functions:
\begin{notimplementedenv}
\begin{listing}{1}
/*@
  @ logic integer \offset_min<a>{L}(a *p);
  @ logic integer \offset_max<a>{L}(a *p);
  @/
\end{listing}
\end{notimplementedenv}

\begin{itemize}
\item \lstinline|\offset_min{L}(p)| is the minimum integer
\lstinline|i| such that \lstinline|(p+i)| is a
  valid pointer at label \lstinline|L|.

\item \lstinline|\offset_max{L}(p)| is the maximum integer
  \lstinline|i| such that \lstinline|(p+i)| is a
  valid pointer at label \lstinline|L|.
\end{itemize}
The following properties hold:
\begin{listing}{1}
\offset_min{L}(p+i) == \offset_min{L}(p)-i
\offset_max{L}(p+i) == \offset_max{L}(p)-i
\end{listing}
It also introduces some syntactic sugar:
\begin{notimplementedenv}
\begin{listing}{1}
/*@
predicate \valid_range<a>{L}(a *p,integer i,integer j) =
  \offset_min{L}(p) <= i && \offset_max{L}(p) >= j;
*/
\end{listing}
\end{notimplementedenv}
and the ACSL built-in predicate \lstinline|\valid{L}(p+(a..b))| is now equivalent to
\lstinline|\valid_range{L}(p,a,b)|.

\subsection{Strings}

\experimental
\begin{notimplementedenv}
The logic function
\begin{listing-nonumber}
//@ logic integer \strlen(char* p);
\end{listing-nonumber}
\end{notimplementedenv}
denotes the length of a 0-terminated C string. It is a total function,
whose value is nonnegative if and only if the pointer in argument is
really a string.

\begin{example}
  Here is a contract for the \lstinline|strcpy| function:
  \listinginput{1}{strcpyspec.c}

\end{example}

\section{Memory leaks}

\experimental

Verification of absence of memory leak is outside the scope of the
specification language. On the other hand, various models could be set
up, using for example ghost variables.

%%% Local Variables:
%%% mode: latex
%%% TeX-PDF-mode: t
%%% TeX-master: "main"
%%% End:
