\subsection{for with range statement [C++]}

\emph{TODO: Grammar additions needed}

\lang{} introduced a new kind of iteration statement: the range-based for statement, illustrated in the code below:
\begin{lstlisting}
void m() {
  int k = 0;
  for (int i: {1,2,3,6,7,8}) {
      k += i;
  }
  
  k = 0;
  std::list<int> data = ...
  for (auto value: data) {
     k += value;
  }
}
\end{lstlisting}

In each case the value of \lstinline|i| or \lstinline|value| is progressively the respective array or list elements. The  
index for the array or iterator for the list that would be used in other forms of \lstinline|for| statements is implicit here, simplifying and
clarifying the code.

However this form poses a difficulty for reasoning about the loop. Typically an iteration over the array would be specified as follows:
\begin{lstlisting}
void m() {
  int k = 0;
  int array[6] = {1,2,3,6,7,8};
  /*@
    loop invariant 0 <= j <= 6;
    loop invariant k == \sum(0,j-1, \lambda int j; array[j]);
    loop variant 6 -j;
  @*/
  for (int j = 0; j < 6; j++) {
    i = array[j];
    k += i;
  }
}
\end{lstlisting}

Note that the loop index, here \lstinline|j|, is needed in the loop invariant to help state the inductive loop invariant. However, there is no explicit loop index in the
for-range syntax. So we define one, named \lstinline|\count|, for specification purposes.
\lstinline|\count| is the number of completed loop iterations so far; 
at the beginning of the initial iteration its value is 0; it 
increments by 1 each time the loop body is started again (including by a
\lstinline|coninue| statement).
We also define \lstinline|\data| to refer to the value of the array or list over which the for-loop ranges.

With these constructs, we can specify our two example uses of for-range as follows:
\begin{lstlisting}
void m() {
  int k = 0;
  /*@
    loop invariant 0 <= \count <= 6;
    loop invariant k == \sum(0,\count-1,\lambda integer j; \data[j]);
    loop variant 6 -\count;
  @*/
  for (int i: {1,2,3,6,7,8}) {
    k += i;
  }

  k = 0;
  std::list<int> data = ...
  /*@
    loop invariant 0 <= \count <= 6;
    loop invariant k == \sum(0,\count-1,
                \lambda integer j; *(\data.begin()+j));
    loop variant 6 -j;
  @*/
  for (auto value: data) {
    k += value;
  }
}
\end{lstlisting}

\emph{TODO: The above may need to be revised when the capabilities for working with iterators are settled.}

