\begin{example}
  The declaration
  \begin{listing-nonumber}
    //@ logic int f(int x) = x+1 ;
  \end{listing-nonumber}
  is not allowed because \lstinline!x+1!, which is a mathematical
  integer, must be casted to \lstinline|int|.  One should write either
  \begin{listing-nonumber}
    //@ logic integer f(int x) = x+1 ;
  \end{listing-nonumber}
  or
  \begin{listing-nonumber}
    //@ logic int f(int x) = (int)(x+1) ;
  \end{listing-nonumber}
\end{example}