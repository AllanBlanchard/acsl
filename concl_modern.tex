
\chapter{Conclusion}

This document presents a Behavioral Interface Specification Language
for ANSI C source code. It provides a common basis that could be
shared among several tools.
The specification language described here is intended to evolve in the
future and remain open to additional constructions.
One interesting possible extension regards ``temporal''
properties in a large sense, such as liveness properties, which can
sometimes be simulated by regular specifications with ghost
variables~\cite{giorgetti06fase}, or properties on evolution of data
over the time, such as the history constraints of JML, or in the Lustre
assertion language.

\oldremark{DP}{
ouverture vers d'autres formalismes de specification: temporal logic, etc.

Donner des elements pour traduire vers ACSL. Utilisation des
variables/champ modeles.

et history contraints de JML ?

Referencer des travaux lies: Leino, Schulte ; Huisman et al.
}

\oldremark{Colaco}{

  Et langage d'assertion a la Lustre ? operateur pre et des langages
  synchrones

  voir ca en terme de flots de valeurs d'une variable a un point
  donne'

  la encore on peut pretendre que l'on peut encoder :


//@ assert P(v->pre(x),x);

devient

//@ ghost static int xpre = v;
//@ assert P(xpre,x);
//@ ghost xpre = x;


Un compilateur comme SCADE devrait engendrer lui meme ce genre d'annotation.


}

\oldremark{SD9}{

  illustrer l'utilisation de ghost pour specifier des proprietes temporelles

}




%%% Local Variables:
%%% mode: latex
%%% TeX-PDF-mode: t
%%% TeX-master: "main"
%%% End:
