\section{Access control and abstraction [C++]}
\label{sec:access}

\TODO{Grammar additions needed}

\subsection{Access control of \NAME{} constructs}

\lang{} allows members of aggregate types to be declared as \lstinline|public|, \lstinline|protected|, or \lstinline|private|. This access control modifier affects whether the member can be accessed within other contexts. 

\NAME{} constructs, on the other hand, are always public and accessible from any other scope (perhaps using scope qualifiers). 
Furthermore, \lang{} constructs are always accessible within \NAME{}
specification constructs.

One could adopt a more restrictive design, in which \NAME{} constructs
obey the same accessibility rules as do \lang{} constructs. 
However, the experience with other specification languages is that 
public accessibility is almost always what is needed. Hence the design
of \NAME{} avoids the complexity of detailed accessibility rules.
%The same access rules apply to \NAME{} constructs declared at class scope. Any name that may be declared at class scope, such as logic functions, predicates, invariants and ghost code, has associated with it the access modifier that it would if it were a \lang{}
%construct, with the same access defaults being used for struct and class declarations. Typically, \lstinline|private| \NAME{}
%constructs would be used to aid in the proofs of the implementation of a class's members, whereas 
%\lstinline|public| \NAME{} constructs are used as class abstractions or to assist in proofs of functions that use the class's
%functionality.

%Recall that a \lang{} context A can declare as a \lstinline|friend| a name N from context B and thereby allow 
%the definition of B::N to use private elements of A. Similarly, within an \NAME{} annotation, context A can declare as
%\lstinline|friend| a \NAME{} name N from context B and thereby allow 
%the definition of B::N (within an ACSL annotation) to use private elements of A.
%
% \TODO{Need an example of access and motivating examples of the rul about friends.}

It follows also then that \lstinline|friend| declarations are not 
needed to provide access to \NAME{} specifications from other classes,
as all specifications are \lstinline|public| by default.
 
\subsection{Access, specifications and information hiding}

The typical purpose of using different access modifiers in a
\lang{} implementation is information hiding: some aspects of
the implementation are not meant to be part of the public API;
by keeping the implementation aspects out of the public API, the
author of the code is free to change the implementation without
affecting the API.

Simply written specifications typically refer to the private
aspects of an implementation and thus can break information hiding.

%If a particular \lang{} name is accessible and that name has an associated \NAME{} specification, then the specification is also accessible and all the names used within the specification must be
%accessible. For example, if a function definition is accessible in some context, then the function contract is also accessible, so all the 
%names used within that function contract must be accessible. This is a conceptual requirement: if a client of a class uses a function of the class, the client needs to be able to see the specification of that function, and thus it must have access to all the identifiers 
%within the expressions of that specification.
%
%However, this naturally expressed requirement causes problems in writing specifications that adhere to the information hiding expectations of classes. 
Consider a class that wraps an implementation of a simple array:

\begin{listing-nonumber}
class Array {
	
  private:
	
	int* data;
	int data_length;

  public:
	
	/*@ 
	requires 0 <= index < data_length;
	assigns \nothing;
	ensures \result == data[index];
	*/
	int getValue(int index) {
		return data[index];
	}
}
\end{listing-nonumber}

Here the representation of the Array object, its data values and length, are appropriately private, while the accessor function is public.
However, expressing the specification of the accessor function requires using the private members, violating information hiding and
the abstraction boundary.

The core problem is this: to express the valid values of \lstinline|index|, one needs the concept of the length of the array; that is, one needs an abstraction, namely length(). In this example the abstract \lstinline|length()| is represented by the data value \lstinline|data_length|, but the class could have chosen some other representation.

One solution to this representation problem is to write the following:

\begin{listing-nonumber}
class Array {
	
  private:
	
	int* data;
	int data_length;
	
  public:
	
	/*@
	   logic int length() = this.data_length; 
	   
	   logic int value(int index) = this.data[index];
	*/
	
	/*@ 
	   requires 0 <= index < length();
	   ensures \result == value(index);
	*/
	int getValue(int index) {
		return data[index];
	}
}
\end{listing-nonumber}

This example demonstrates the general solution for abstraction that respects information hiding: write logic functions, predicates or invariants that express the abstract properties of the class, express the specifications of the public members of the class in terms of those (public) abstractions, and write the implementation of the abstractions in terms of the appropriate private aspects of the class.

%\subsection{Usability improvements}
%
%In the example above we abstracted the concept of array length by defining the public logic function \lstinline|length()|. 
%In this case, the sole purpose of \lstinline|length()| is 
%to hide the implementation field \lstinline|data_length|. It may seem verbose and overkill to need to define and use \lstinline|length()|  solely to expose \lstinline|data_length|. An alternative is to declare
%that although \lstinline|data_length| is private for \lang{} implementation purposes, it is public for specification purposes.
%In this case we write
%
%\begin{listing-nonumber}
%class Array {
%	
%  private:
%	
%    //@ spec_public
%  	int[] data;
%	
%    //@ spec_public
%	int data_length;
%	
%  public:
%	
%	/*@
%	   requires 0 <= index < this.data_length;
%	   logic int value(int index) = this.data[index]; 
%	*/
%	
%	/*@ 
%	   requires 0 <= index < data_length;
%	   ensures \result == this.value(index);
%	*/
%	int getValue(int index) {
%		return data[index];
%	}	
%}
%\end{listing-nonumber}
%
%The \lstinline|spec_public| access specifier applies only to the next declaration within the class declaration: in the example above both
%\lstinline|data| and \lstinline|data_length| are still 
%\lstinline|private| in \lang{}, since that is the most recent \lang{} access specifier, but the declarations have \lstinline|public| access within specifications. 
%
%The advantage of this syntax is more concise expression; there is no need to declare \lstinline|length()| just to replicate \lstinline|data_length|.
%The disadvantage is that we have lost a measure of information hiding. By using the abstraction \lstinline|length()| we could
%have altered the representation of length within the class without affecting the public specifications and what clients of the class see.
%By using \lstinline|spec_public|, that is no longer the case.
%Accordingly, \lstinline|spec_public| should only be used in temporary situations, during development, or where it is clear that the representation is not going to change.
%
%The access specifier \lstinline|spec_protected| may be used (within \NAME{} annotations) to give \lang{} \lstinline|protected| access to \lang{} constructs
%that would otherwise be \lstinline|private|.

\subsection{Pure functions}
\label{sec:pure}
A verbose aspect of the listing above is that the logic representation function \lstinline|value()| is essentially a replica of
the \lang{} function \lstinline|getValue()|. Why write and maintain both of these? Can we not use at least some \lang{} functions in 
specifications?

The syntax to do so is shown in the following listing, but there are restrictions on when this is allowed.

\begin{listing-nonumber}
class Array {
	
  private:
	
    int* data;
    int data_length;
	
  public:
	
    /*@ 
      requires 0 <= index < data_length;
      ensures \result == data[index];
      pure
    */
    int getValue(int index) {
		return data[index];
    }	
}
\end{listing-nonumber}

When designated \lstinline|pure|, a \lang{} function may also be used in a \NAME{} specification. The conditions under which a 
\lang{} function may be declared \lstinline|pure| are discussed in the following subsections.

\subsubsection{No side-effects}
The principal restriction is that the function declared \lstinline|pure| must have no side-effects. It may compute a value and may have 
temporary stack variables, but it may not assign to any locations in the heap that are allocated in the pre-state of the function.
No assignments are permitted to a pre-state location even if the assignment does not change the value stored in the location.
This is partly a static condition, but when assigning to fields of an object, it must be demonstrated that the object is a local temporary or
has been allocated since the pre-state; this demonstration may require reasoning about aliasing.

\subsubsection{No heap changes}
The rule prohibiting side-effects applies to the heap as well:
no allocations or deallocations of memory are allowed in pure functions.
Stack operations are allowed. Thus, for example,
 implicit copy constructors that 
only create and destroy stack temporaries are permitted.

\subsubsection*{Reading the heap}
Since the pure function is indeed a C++ function, it may include read operations on the heap. When the pure function is used as a logic function,
it must be invoked with respect to a given heap state. 
That is, the pure function has a single implicit state label.
An example showing this use is given below:
\begin{lstlisting}
class C {
private:
  int count;

public:
   /*@ ensures \result == count;
       pure
     @*/   
  int getCount() { return count; }
  
  
  void test() {
      x:
      count++;
      //@ \textsl{}assert getCount{Here}() == 1 + getCount{x}();
  }

\end{lstlisting}

A special case of pure functions are those declared in \lang{} as \lstinline|constexpr|.
Such functions compute constant values that may be evaluated at compile time. A function declared \lstinline|constexpr| is implicitly \lstinline|pure| and my be used in a logic expression.

	\emph{Is the rule concerning strong purity too draconian?}
