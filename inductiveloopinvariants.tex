The program
\begin{c}
  int x = 0;
  int y = 10;

  /*@ loop invariant 0 <= x < 11;
    @*/
  while (y > 0) {
    x++;
    y--;
  }
\end{c}
is not correctly annotated, even if it is true that x remains less
than 11 during the execution. This is because it is not true that the
property $x<11$ is preserved by the execution of \verb|x++ ; y--;|. A
correct loop invariant could be \verb|0 <= x < 11 && x+y == 10|, true
at loop entrance and preserved (under the assumption of the loop
condition \verb|y>0|).

Similarly with general invariants: in
\begin{c}
  int x = 0;
  int y = 10;

  while (y > 0) {
    x++;
    //@ invariant 0 < x < 11;
    y--;
    //@ invariant 0 <= y < 10;
  }
\end{c}
the propositions given as invariants are correct assertions but
not inductive invariants: \verb|0 <= y < 10| is not a consequence of
hypothesis \verb|0 < x < 11| after executing \verb|y--| ; and 
\verb|0 < x < 11| cannot be deduced from \verb|0 <= y < 10| after looping back
through the condition \verb|y>0| and executing \verb|x++|. Correct
invariants could be:
\begin{c}
  while (y > 0) {
    x++;
    //@ invariant 0 < x < 11 && x+y == 11;
    y--;
    //@ invariant 0 <= y < 10 && x+y == 10;
  } 
\end{c}
