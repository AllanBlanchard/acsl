 
\chapter{Examples}

Consider the application of the transform operation to a unary
operator with side effects.

\lstinputlisting{transform.cpp}

Functors are applied using their \texttt{operator()} member function. 
In general we do not statically know what functor is being used as the actual argument.
So we need to be able to express some specification properties.
\begin{itemize}
\item \textbackslash pre(f,i) -- a predicate that is true when the precondition of f(i) is true
\item \textbackslash post(f,r,i) -- a predicate that is true when the postcondition of f(i) with result value r is true
\item \textbackslash assigns(f,i) -- \textbackslash locset that contains any memory location assigned when f(i) is called with argument i
\end{itemize}
So for the value \texttt{op} in the code above.
\begin{itemize}
	\item \lstinline|\pre(op,i) == \true|  (ignoring any overflow checks)
	\item \lstinline|\post(op,r,i) == (r == -i && sum == \old(sum + i))|
	\item \lstinline|assigns(op,i) == { sum }|
\end{itemize}

Now \lstinline|transform| applies \lstinline|op| repeatedly to the elements of \lstinline|vector|. 
If \lstinline|op| were a simple function of its input only with no side effects, we could write a postcondition easily as something like
\lstinline|(\textbackslash forall int i; 0 <= i <= a.size(); vector[i] == op(\old(vector[i]))|.

However, <code>op</code> may easily be more complex than that. 
It may have more complex inputs and it may have side-effects. 
In general we cannot know the effect of repeated application of <code>op</code> without simulating the loop implicit in <code>transform</code> and reasoning about it with inductive invariants.

The specification of \lstinline|transform| could look like this.
\lstinputlisting{transform_spec.cpp}

Here the block within the specification, enclosed in curly braces, is
a \textit{model program}.  A model program specifies the behavior of
some method by writing abstracted code (pseudo-code, if you will) that
summarizes the method's behavior.  For small programs, this might
simply nearly replicate the actual implementation.  In that case,
reasoning about uses of the method are tantamount to inlining the
implementation.

The implicit (ghost) variable \lstinline|\count| denotes the number of completed iterations of the loop, beginning with 0. 
For a simple loop index that counts up by 1 from 0, the loop index is equal to \lstinline|\count|. 
For more complex loops and loops using iterators, 
\lstinline|\count| is useful in writing specifications.

Now we need to include the effects of \lstinline|op| in the above specification of \lstinline|transform| without knowing what \lstinline|op| actually is. We do so in terms of its own specification.

A snippet of code like \lstinline|y = op(x);| can be summarized as
\begin{listing-nonumber}
	assert \pre(op,x);
	havoc \assigns(op,x);
	int temp_result;
	assume \post(op,temp_result,x);
	y = temp_result;
\end{listing-nonumber}

Putting that snippet together with the partial specification of 
\lstinline|transform| gives us this.

\lstinputlisting{transform_spec2.cpp}

Although, not essential in this case, in general we separate subexpressions by evaluating each subexpression to a temporary intermediate result. That way any implicit conversions and questions of well-definedness can be made explicit with appropriate assertions.
\section{Misc stuff}


