\section{A design for functional programming in C and \lang}
\label{sec:functionalDiscussion}

This section motivates the design of functional programming features in \NAME, describes how such features might be embedded in a deductive verification system, and provides several examples and use cases.

The first few sections summarize the use of function pointers
in C/C++ and give a simple model for basic reasoning. 
The crux of the design is specifying and reasoning about application of 
function pointer values; this is addressed in the bulk of this
section.
Two mechanisms are used to attach specifications to function literals: conventional function contracts for classic C/C++ functions and 
specification interfaces (\S\ref{sec:lambdaspecs}) for lambda expressions; 
reasoning about functional objects is accomplished using
dynamic contracts (\S\ref{sec:fpreasoning}). 
These mechanisms are then demonstrated with examples and use cases in \S\ref{sec:fpexamples}ff.

 
\subsection{Functional programming features in C and \lang}
\label{sec:fp}

C programs allow creating function-pointer values that 
can be used in expressions and assignments and later applied
to a set of arguments to yield a result. 
\lang adds 
the	ability to take pointers to functions that are elements
of classes and other aggregates and to create functions
without the need to declare the function (lambda expressions). 
Correspondingly \acsl and \NAME need to be able to manipulate such objects and reason about their effects.

Functional objects (including what are typically called function pointers) in C and \lang are subject to the following operations:
\begin{itemize}[noitemsep,nolistsep]
	\item creation
	\item assignment and initialization
	\item address-of operation
	\item casting and conversion
	\item equality comparison
	\item application
\end{itemize}

Functor objects are an alternative to simple function pointers.
Functor objects are instances of classes that contain an 
\lstinline|operator()|. 
They can be called as shown at the end
of the code in Example \ref{Ex-FN1}. 
The relationships between
`naked' function pointers and functor objects and techniques for 
specifying and reasoning about functor objects are presented in \S\ref{sec:functors}.

\subsubsection{Creation}

Function objects can be created by
\begin{itemize}[noitemsep,nolistsep]
	\item Taking the address of a textually defined global function.
	\item Taking the address of a member function of an aggregate.
	\item Writing a lambda expression.
\end{itemize}
These are demonstrated in Example \ref{Ex-FN1}.
Using a \lstinline|&| to convert a function name to a function
pointer is optional for static functions. The conversion is performed implicitly where needed. The \lstinline|&| is needed when taking the address
of a member function.
\begin{example}
	 \label{Ex-FN1}
	Example uses of C and C++ function values.
	\listinginput{1}{fexample1.cpp}
\end{example}

Function pointers can be taken from inline functions. In that case 
the actual application of the function will not be inlined: a 
non-inlined instance of the function will be created by the compiler.

The type of functions with default arguments includes all the arguments; any defaults are ignored.

\subsubsection{Assignment, initialization and typing}
A function pointer type includes the type and number of its arguments,
the type of the return value, and any qualifications (such as
\lstinline|const|).  In addition, whether the final argument is a
var-args argument is also part of the type.  The type of a pointer to
member function includes the type of the object it is called for.

The declaration of a variable with function pointer type is given by syntax like\\
\emph{ret-type} (*\emph{var-name})(\emph{arg-types}), with optional \lstinline|const| and other modifiers.

A variable with a function pointer type can be initialized with or assigned a function pointer value if they have the same type.

A \emph{static} member function and
a global function with the same signature have the same type.
\emph{Non-static} member functions have different types because
the signature and type implicitly include the object the member function
is called for and the fact that it is a member function.

A lambda expression \emph{that has an empty capture specification} and
a global function with the same signature have the same type.
Lambda expressions with non-empty capture specifications are discussed 
in \S\ref{sec:functors}.

\subsubsection{Address-of}

A function pointer is a pointer, to a function. 
The \lstinline|&| operator is traditionally used to take
the address of something. 
This operation is shown in Example \ref{Ex-FN1}.
However, there is an automatic conversion from the textual
name of a function to function pointer value. 
So the use of \lstinline|&| is optional for static functions; it is required for member function pointers.

\subsubsection{Casts and conversions}

\textit{To be written}
\TODO{TODO: Casts and conversions}

\subsubsection{Comparison}

The only comparison operations on function pointers are 
equality and inequality. Two function pointers of the 
same type are equal precisely when they point to the same
textual definition; since functions must have only one
definition (though perhaps multiple declarations), 
the instance being referred to is unique. 
The equality operation is type-incorrect if the two
pointers have different types.

Two function pointers are not equal if they point to 
different textual functions even if they implement precisely the
same computation or have precisely the same text.

Function pointers can also be compared to the predefined value \lstinline|nullptr|. A function pointer variable may have
a value of \lstinline|nullptr|, but the value of the address
of any textual function or member function or lambda expression
 is always not
equal to \lstinline|nullptr|.

\subsubsection{Application}

The value of a function pointer is typically known only by
executing the program. (Otherwise, one could simply call the
known function directly.) To call the function to which the
function pointer points, one uses this syntax:
(\emph{expr})(\emph{args}) where \emph{expr} is an expression yielding a value of the correct type for the argument list. If the expression is
more than a simple name, parentheses are required.
Very typically the expression is simply the name of a variable of function pointer type, or *\emph{name} where the optional \lstinline|*| indicates a dereference (the dereference is always implicitly applied).
In this case, with the name being \lstinline|fp|, the syntax is simply (fp)(\emph{args}) or just
fp(\emph{args}).

Note that when a function is called through a function pointer, all arguments must be present; the function pointer knows nothing about
default arguments.

When calling member functions through member function pointers, one must supply an object along with the actual arguments. The syntax is \\
(\emph{object}.*\emph{mfp})(\emph{args}) or
(\emph{object-ptr}->*\emph{mfp})(\emph{args}).
Note that dynamic dispatch is still used for virtual functions,
as in the following example:
\lstinputlisting{virtual_fct_ptr.cpp}

\subsection{Modeling function pointers in logic}
\label{sec:modelingfp}
 
 Since function pointers are unique values with no constituent parts, they can be modeled in logic simply as a set of distinct values. It simplifies the reasoning engine's task if
 function pointers are considered distinct from other objects.
 Conversion to or from other objects is rare, but, if necessary can still be accomplished using logical adapter functions.
 
 In this model
 \begin{itemize}
 	\item Each textual instance of a function, member function or lambda expression is assigned a unique logical literal.
 	Since the programming language takes care of type checking,
 	all of these literals can be of the same logical sort.
 	\item Function pointer values consist of one of these logical literals or a logical value corresponding to 
 	\lstinline|nullptr| for function pointers.
 	\item \TODO{ about conversions }
 \end{itemize}
The above model is sufficient for reasoning about initialization, assignment, conversions, and comparisons.

\subsection{Specifications for lambda expressions}
\label{sec:lambdaspecs}

Conventional function definitions have an established place to insert their specification (cf. \S\ref{sec:fn-behavior}). Functions defined by lambda expressions do not. These appear in the middle of expressions where the insertion of a function contract would be intrusive and difficult to read.

C++ represents lambda expressions as an \lstinline|operator()| wrapped in an anonymous unique class. We can follow this 
approach in providing a function contract for a lambda expression.  
For example, if we have a lambda expression such as \lstinline|[](int x, int y)-> { return -x; }| we can write its contract like this.

\lstinputlisting{operator.cpp}

We call such a class a \emph{specification contract class}.
We use a \emph{specification cast} to attach this contract to
the lambda expression:\\
\lstinline|std::foreach(first, last, /*@ (Negator)@*/[](int x, int y)-> { return -x; }|

For a simple lambda expression such as the one above, a tool might readily infer the necessary specification contract class and cast. But lambda expressions can be arbitrarily complex, in which case a user-written specification contract class and cast will be necessary.

Casts such as these create a proof obligation, namely, that the 
lambda expression does indeed satisfy the specification given in the
specification contract class.

\subsection{Reasoning about function pointer application}
\label{sec:fpreasoning}
The meat of reasoning about function pointers is reasoning about the result of application of an unknown pointer to a set of arguments. 
The properties of a function are its
preconditions, postconditions, frame-conditions and other effects on the program state. 
For textually defined functions, these properties are given by 
the function contract, which is textually written in
conjunction with the function definition or declaration.
We need a comparable means to express the contract of an
arbitrary unknown value of a given function pointer type.

Consider preconditions. 
A precondition is a predicate over a
function's arguments and possibly the program state; it is 
true when a call of the function with those arguments is 
well-defined (that is, is known to obey the rest of the function contract). 
For a function pointer \emph{value}, say \lstinline|fp|, we can express this predicate as a logical function
\lstinline|\requires(fp,|\emph{args}\lstinline|)|. 
Furthermore, we can implicitly define this term for every function in the program -- or at least for every function in the program whose address is taken and thus might be the value of a function pointer expression.

Note that \lstinline|\requires| is not a term itself. It is the name of a family of logic functions that differ by the types of the function arguments. There is in addition a state label representing the program state, in order to capture
information about values used by the function that are not in
the argument list. 

There is a corresponding family of terms or predicates for each kind of function contract clause:
\begin{itemize}[noitemsep,nolistsep]
	\item \lstinline|\requires|, returning a \lstinline|boolean| value
	\item \lstinline|\ensures|, returning a \lstinline|boolean| value
	\item\lstinline|\assigns|, returning a \lstinline|tset| value
	\item \textit{Other clauses to be completed} \TODO{ allocates, frees, throws, terminates, decreases, exits, breaks, continues, returns}
\end{itemize}

Note the following:
\begin{itemize}
	\item Some clauses, such as the \lstinline|ensures| clause,
	state properties comparing two states. They therefore have two state labels.
	\item The property represented by these terms is that
	given by the effective clause value combined across all the
	behaviors of the function contract. Thus there are no
	corresponding dynamic predicates for the \lstinline|assumes| clause (see the example below).
	\item Some clauses allow using special terms (such as 
	\lstinline|\result|, 
	\lstinline|\exception|, \lstinline|\exit_status|). 
	These are represented by a special explicit argument. Consider, for example, the ensures predicate for the example below.

\begin{listing-nonumber}
int j;
		
/*@
requires r0;
behavior b1l:
  assumes a1;
  requires r1;
  ensures \result == \old(j) + k;
behavior b2:
  assumes a2;
  requires r2;
  ensures \result >= 0;
*/
int m(int k) { ... }
\end{listing-nonumber}

	In this case the consolidated precondition is the following:
	\begin{listing-nonumber}
          r0 && (a1==>r1) && (a2 ==> r2)
        \end{listing-nonumber}
	while the consolidated postcondition is:
	\begin{listing-nonumber}
          (a1==>(\result==\at(LPre,j)+k)) && (a2 ==> (\result>= 0))
        \end{listing-nonumber}
	The C++ function takes a single argument and produces a single result. So its signature is
	\lstinline|\ensures{LPre,LPost}(int (*fp)(int), int result, int k)|	
	and its definition for the function \lstinline|m| (presuming the
        function has a logical literal identifier \lstinline|M|) is:
	\begin{listing-nonumber}
        \ensures{LPre,LPost}(M,r,k) <==>
            (r0 && (a1==>(r==\at(j,LPre)+k)) && (a2 ==> (r >= 0))
        \end{listing-nonumber}
	The postcondition term is true for any combination of 
	result value and argument values for which the ensures clause would be true. Any underspecification of the function is readily represented.

\end{itemize}

\subsection{Member function pointers}
\label{sec:memberfp}
Member function pointers occupy a different region of type space than
do static function pointers, but are nevertheless treated quite
similarly. Each member function has implicitly provided the same
\lstinline|\requires| and other dynamic specifications. Now however
the definition of those predicates can include references to
\lstinline|this| to refer to a pointer to the object the member
function is called for. If we take the following example:

\lstinputlisting{member_fct_ptr.cpp}

we have (with \lstinline+|A::m|+ the name of the literal for the textual
function) the implicitly defined terms

\begin{lstlisting}
  \forall A t; \requires(|A::m|,t) == this->x > 0;
  \forall A t; \assigns(|A::m|,t) == \empty;
  \forall A t, int result; \ensures(|A::m|,t,result) == (result == t->x + 1);
\end{lstlisting}
(Here the quantification variable \lstinline|t| ranges over the possible
objects the member function is called for.)

\subsection{Examples}
\label{sec:fpexamples}

This section presents a number of manually worked examples 
to demonstrate how the design above enables reasoning about
a variety of uses of function pointers.
In these examples we ignore questions of arithmetic overflow and 
differences between machine integers and mathematical integers.

The examples show a snippet of C++ code and then a translation into a logical
pseudo-code. The pseudo-code contains declarations, assignments, state labels, assume, assert and havoc statements. 
The assume statements are axioms implied by the semantics of the
programming language, i.e. C++. 
The assert statements are verification conditions that
must be proved according to the rules of \NAME.
Havoc statements give the listed memory locations arbitrary values (which are then constrained by the postcondition). Also,
\begin{itemize}[noitemsep,nolistsep]
	\item the symbol \lstinline|\flit| represents the logical sort of function pointers;
	\item the symbol \lstinline|<:=| is the relation between two sets of memory locations that the left side is a subset of the right side;
	\item the \lstinline|@| symbol is an operator giving the value of a memory location (the lhs) at a given program state (the rhs).
\end{itemize}

We also use arithmetic, logical, and quantified expressions in what should be familiar syntax to ACSL and C++ knowledgeable readers.

It will be helpful in reading the logical representation to recall that
a translation of a function to be verified consists of
\begin{itemize}[noitemsep,nolistsep]
	\item declarations of formal arguments, external global variables, and a special value for the return value of the function being proved
	\item assuming the preconditions of the function
	\item translating the body of the function
	\item asserting (to be proved) the postconditions of the function
\end{itemize}
A function call within the body of the function being verified is
translated as follows:
\begin{itemize}[noitemsep,nolistsep]
	\item declaration of a temporary variable to hold the result of the subexpression that is the function call
	\item assertion (to be proved) that the precondition of the callee is satisfied
	\item assertion (to be proved) that the frame of the callee (the set of memory locations possibly written by the callee) is a subset of the frame of the caller
	\item havoc (set to arbitrary values) any memory location in the frame of the callee
	\item assume the postcondition of the callee
\end{itemize}

For the first few examples, the (manual) translation into SMT-LIB is shown.

\subsubsection{Simple initialization and application}

Consider the code in the following listing. It initializes a function pointer with a known value and then applies it to an argument.
\lstinputlisting{fexampleA.cpp}

First note that if a tool processing this code performed constant propagation and constant folding, the tool would realize, as a human immediately does, that the function 
pointer being applied in line 3 of \lstinline|main| can only point to 
\lstinline|increment|, and so we can use \lstinline|increment|'s specifications directly.
For the sake of generality, though, we will presume that such 
optimizations are not available.

The translation into pseudo-code gives
\lstinputlisting{fexampleA.logic}

Here
\begin{itemize}[noitemsep,nolistsep]
	\item Line 1 gives a declaration of the logic name representing 
	the function pointer literal for the \lstinline|increment| function
	\item Lines 2--5 show the properties of this literal obtained
	automatically from \lstinline|increment|'s function contract
	(Lines 1-3 in the C++ code).
	\item Line 6 declares a logic name for the program variable 
	\lstinline|fp|; line 6 is the translation of line 7 of the C++ code above.
	\item Line 8 is a state label marking the state prior to the 
	function call (\lstinline|fp(4)|).
	\item Line 9 is an assertion of \lstinline|fp|'s precondition
	\item The check on the frame condition is omitted
	\item Line 10 havocs the values in the frame of \lstinline|fp|,
	whatever they might be (here the expression evaluates to 
	\lstinline|\empty|).
	\item Line 11 declares a temporary variable to hold the result of
	the subexpression that is the function call
	\item Line 12 gives a state label for the post-state of the function call
	\item Line 13 is the assumption of the callee postcondition. It relates the two labeled states --- before and after the function call.
	\item Line 14 is the declaration of a logic variable corresponding 
	to the program variable of the same name
	\item Line 15 is the translation of the assignment of the result of
	the function call to the program variable \lstinline|y|.
	\item Line 16 asserts the postcondition to be proved
\end{itemize}

The corresponding SMT proof obligation is
\lstinputlisting{fexampleA.smt2}
which is proved immediately.

There a few points to note here:
\begin{itemize}
	\item Because \lstinline|\requires| denotes a family of logic functions, one for each different signature, we represent them in SMT-LIB with mangled names.
	\item Since there are no state changes in this example, we do not introduce variable incarnations as needed by the traditional 
	single assignment form of SMT-based verification conditions.
\end{itemize}

\subsubsection{Function pointer expression}

This next example slightly expands on the former example: now the function pointer is initialized with an expression and is not a constant.
\lstinputlisting{fexampleB.cpp}

The translation into pseudo-code gives
\lstinputlisting{fexampleB.logic}

The corresponding SMT-LIB is
\lstinputlisting{fexampleB.smt2}

\subsubsection{State changing application}
The next example requires using labels to distinguish program states.
\lstinputlisting{fexampleC.cpp}

The translation into pseudo-code gives:
\lstinputlisting{fexampleC.logic}

Here since the functions and function pointers have a void return type, there is no argument of the \lstinline|\ensures| term corresponding to the return value. Also, the \lstinline|test| function has no frame condition, 
so it is implicitly assumed to be \lstinline|assigns \everything;|, meaning writes to any memory location are allowed. This is evident on
Lines 13 and 16; these assertions state that the callee frame must be 
a subset of the caller's frame, which is \lstinline|\everything|;
the assertions are trivially true. 
\begin{itemize}
	\item Line 17 states that the input argument \lstinline|x| of \lstinline|doTwice|
	must satisfy (a) the precondition of the first application of
	\lstinline|fp|, and then (b), whatever result (\lstinline|t|) might be allowed
	for that first application of \lstinline|fp| must satisfy the
	precondition for the second application of \lstinline|fp|.
	\item Line 18 states that the overall frame of \lstinline|doTwice| is
	the union of the frame for \lstinline|fp| applied to the input and
	\lstinline|fp| applied to the intermediate value.
	\item Line 19 states the postcondition: a value \lstinline|r| is 
	permitted as a result of \lstinline|doTwice| if there is some
	intermediate value \lstinline|t| such that \lstinline|t| is a permitted result of a (first) application of \lstinline|fp| with \lstinline|x| as input and that \lstinline|r| is a permitted result
	of a (second) application of \lstinline|fp| with \lstinline|t| as input.
\end{itemize}
The frame conditions here are \emph{dynamic}: they are not 
necessarily just a simple list of statically known memory locations. The logical encoding and
reasoning about dynamic frames is more difficult than for static frames.
The complexity of specifying and proving frame conditions quickly encourage one to use, as much as possible, functions without side-effects,
that is, to adopt a functional programming style.


\subsubsection{Function pointers as function arguments}
The next example shows the use of function pointers as arguments to other functions: the functions \lstinline|doOnce|
and \lstinline|doTwice| each take a function pointer with signature \lstinline|(int)->int| and apply it once or twice to an
argument. Then the function \lstinline|test| exercises these combinator functions. All three functions need to be verified.

\lstinputlisting{fexampleD.cpp}

This example, although still simple, shows that the dynamic contracts
are beginning to look complicated; they essentially need to account
for intermediate values that are computed in the function body and used as input to subsequent methods.


 
The logic translation of \lstinline|doOnce| is shown next. Its proof is fairly obvious because the assertions to be proved match 
assumptions given.

\lstinputlisting{fexampleD1.logic}

The logic translation of \lstinline|doTwice| is more complicated.

\lstinputlisting{fexampleD2.logic}

\TODO{Comments on assigns reasoning; pure functions; need translation of test methods}

\subsubsection{Composition}

\TODO{}

\subsubsection{Library methods: all\_of}
\label{sec:libraryMethods}
The C++ standard library includes a number of functions that take function pointer or similar arguments and apply them to various 
collections or iterator-defined sequences of values. 
It is important
that reasoning about such constructs be straightforward within \NAME and
associated tools. 

This subsection begins a series of examples showing how such methods might be 
specified and how a reasoning tool might translate and verify uses of such methods. So as to avoid needing to model iterators in general, the 
following examples will demonstrate our function pointer design just
for arrays.

The first such example is the library function \lstinline|std::all_of|,
which operates on a sequence (of array elements, here) to determine
whether a given predicate is true for all of the elements. In this
case the predicate is required to have no side-effects.

\TODO{}

\subsubsection{Library methods: transform}

\TODO{}

\subsubsection{Library methods: foreach}

\TODO{}

\subsection{Functors and capturing lambda expressions}
\label{sec:functors}

An alternate to function pointers and simple lambdas are functors. Functors are classes, typically with state, and an
\lstinline|operator()|. Functors can be used in functional programming environments, in which case it is the special 
\lstinline|operator()| that is called. For example, an instance of a functor class can be passed to 
\lstinline|std::all_of|; the \lstinline|bool operator(T t)| is called on each element of the container to determine if
the operator answers \lstinline|true| for all of them.

Since functors are classes, there is no problem with writing a function contract in place.

\lstinputlisting{ffunctorA.cpp}

The \lstinline|std::any_of| algorithm takes a function object (pointer or functor) as its last argument. To reason about
the effect of \lstinline|std::any_of| requires the dynamic contract predicates to be defined. These can be defined for 
\lstinline|FF.operator()| by direct (and automatic) inspection of the contract for that function.

A lambda expression that does not capture any variables from its scope is equivalent to a function pointer.
A lambda expression
that does capture variables is equivalent to a functor. 
However, as discussed earlier, there is no natural syntactic place to
put a function contract for the lambda expression. We use the same solution as above, namely specification contract classes and casts.
A specification class for a lambda expression looks exactly like a functor declaration. 
The example just above can be rewritten using lambdas and specification classes as follows:

\lstinputlisting{ffunctorB.cpp}

Now the logical encoding of the call of \lstinline|any_of| considers its last argument to have type \lstinline|FF|,
and represents that last actual argument with a function object literal, the properties of which are expressed with 
dynamic function contract predicates.

\subsection{std::function}
\label{sec:stdfunction}

\lstinline|std::function| is a C++ template adapter that can hold any object that takes arguments and returns a value: function pointer, 
functor,
capturing or non-capturing lambda expression,
pointers to member functions, 
and even pointers to member data items.
 \lstinline|std::function|  provides a uniform way to hold, assign, and pass any of these functional objects to generic algorithms.
 
 \lstinline|std::function| is also then a unifying way to reason about all of the functional objects and their uses. In particular,
 \begin{itemize}[noitemsep,nolistsep]
 	\item each static function definition and lambda expression is
 	represented as a logical literal of a sort that is used for all
 	\lstinline|std::function| types.
 	\item each literal has associated with it dynamic function contracts as have been illustrated in past sections
 \end{itemize}

\subsection{Concepts}
\label{sec:concepts}
Although \lstinline|std::function| is the unifying type for functional objects, the direction of C++ is to declare functional 
arguments to standard library algorithms using \emph{Concepts}.
There are compile-time evaluated restrictions on class objects (such as
functors) that can impose such restrictions as having boolean member of function of a given signature, or being copy-constructible and the like.
Concepts are part of C++ as of C++20 and so won't be considered 
at present 
for \NAME.

\TODO{TODO: what about quantification over labels}

\TODO{TODO: Check all uses of labels in the examples}
