\subsection{Parsing annotations in practice}

In the original (University of Iowa) JML tools, parsing was done by simply ignoring \verb|//@|, \verb|/*@| and
\verb|*/| at the lexical analysis level. This technique could modify the
semantics of the code, for example:
\begin{listing}{1}
return x /*@ +1 */ ;
\end{listing}
In our language (as in the definition of JML and current JML tools, such as OpenJML), this is forbidden. Technically, the current
implementation of Frama-C isolates the comments in a first step of
syntax analysis, and then parses a second time. Nevertheless, the
grammar and the corresponding parser must be carefully designed to
avoid interaction of annotations with the code. For example, in code such as
\begin{listing}{1}
  if (c) //@ assert P;
     c=1;
\end{listing}
the statement \lstinline|c=1| must be understood as the
branch of the \texttt{if}. This is ensured by the ACSL grammar,
which states that \lstinline|assert| annotations are not statements themselves,
but attached to the statement that follows, like C labels.
