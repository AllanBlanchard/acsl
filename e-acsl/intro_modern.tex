%; whizzy-master "main.tex"
\chapter{Introduction}

This document is a reference manual for
\ifthenelse{\boolean{PrintImplementationRq}}%
{the E-ACSL implementation provided by the Frama-C
  framework~\cite{frama-c}.}%
{E-ACSL.}
E-ACSL is an acronym for ``Executable ANSI/ISO C
Specification Language''. It is an ``executable'' subset of
ACSL~\cite{acsl} implemented in the \framac platform~\cite{framac}. Contrary to
ACSL, each E-ACSL specification is executable: it may be evaluated at runtime.

In this document, we assume that the reader has a good knowledge of both
ACSL~\cite{acsl} and the ANSI C programming language~\cite{KR88,standardc99}.

\section{Organization of this document}

This document is organized in the very same way that the reference manual of
ACSL~\cite{acsl}.

Instead of being a fully new reference manual, this document points out the
differences between E-ACSL and ACSL. Each E-ACSL construct which is not
pointed out must be considered to have the very same semantics that it ACSL
counterpart.

\section{Generalities about Annotations}\label{sec:gener-about-annot}
\nodiff

\section{Notations for grammars}
\nodiff
