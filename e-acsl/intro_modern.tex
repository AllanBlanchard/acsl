%; whizzy-master "main.tex"
\chapter{Introduction}

This document is a reference manual for
\ifthenelse{\boolean{PrintImplementationRq}}%
{the \eacsl implementation provided by the \eacsl plug-in~\cite{eacsl-plugin}
  (version\eacslversion) of the \framac framework~\cite{framac}.}%
{E-ACSL.}
\eacsl is an acronym for ``Executable ANSI/ISO C
Specification Language''. It is an ``executable'' subset of
\emph{stable} \acsl~\cite{acsl} implemented~\cite{acslimplem} in the \framac
platform~\cite{framac}. ``Stable'' means that no experimental \acsl feature is
supported by \eacsl. Contrary to \acsl, each \eacsl specification is
executable: it may be evaluated at runtime.

In this document, we assume that the reader has a good knowledge of both
ACSL~\cite{acsl} and the ANSI C programming language~\cite{standardc99,KR88}.

\section{Organization of this document}

This document is organized in the very same way that the reference manual of
\acsl~\cite{acsl}.

Instead of being a fully new reference manual, this document points out the
differences between \eacsl and \acsl. Each \eacsl construct which is not pointed
out must be considered to have the very same semantics than its \acsl
counterpart. For clarity, each relevant grammar rules are given in BNF form
in separate figures like the \acsl reference manual does. In these rules,
constructs with semantic changes are displayed in \markdiff{blue}.

\ifthenelse{\boolean{PrintImplementationRq}}{%
Not all of the features mentioned in this document are currently
implemented in the \framac's \eacsl plug-in. Those who aren't yet are signaled
as in the following line:
\begin{quote}
\notimplemented[Additional remarks on the feature may appear as footnote.]%
{This feature is not currently supported by \framac's \eacsl plug-in.}
\end{quote}

As a summary, Figure~\ref{fig:notyet} synthetizes main features that are not
currently implemented into the \framac's \eacsl plug-in.
\begin{figure}[htbp]\label{fig:notyet}
  \begin{center}
    \begin{tabular}{|l|l|}
      \hline
      typing 
      & mathematical reals \\
      \hline
      terms
      & \lstinline|\\true| and \lstinline|\\false| \\
      & bitwise operators \\
      & let binding \\
      & typeof \\
      & t-sets \\
      & \lstinline|base\_addr|, \lstinline|offset| and
      \lstinline|block\_length|
      \\
      \hline
      predicates 
      & exclusive or operator \\  %     \lstinline|^^|
      & let bindings \\
      & quantifications over non-integer types \\
      & \lstinline|valid| and \lstinline|valid\_range| \\
      & \lstinline|initialized| and \lstinline|specified|
      \\
      \hline
      annotations 
      & behavior-specific annotations \\
      & loop annotations \\
      & global annotations
      \\
      \hline
      behavior clauses
      & assigns \\
      & decreases \\
      & abrupt termination \\
      & complete and disjoint behaviors
      \\
      \hline
    \end{tabular}
  \end{center}
  \caption{Summary of not-yet-implemented features.}
\end{figure}
}%
{}

\section{Generalities about Annotations}\label{sec:gener-about-annot}
\nodiff

\section{Notations for grammars}
\nodiff
