\chapter{Introduction}

This document is a reference manual for ACSL, an acronym for ``ANSI C
Specification Language''. This is a specification language proposed by
the CAT project, implemented in the Frama-C platform, which aims at
specifying behavioral properties of C source code. The main
inspiration for this language comes from the \emph{Java Modeling
  Language} (JML) which aims at the same goals for Java source code,
which itself is inspired from the \emph{design-by-contract} principle
proposed by Bertrand Meyer\footnote{Who indeed took its own
  inspiration from the concepts of preconditions and postconditions on
  a routine, go back at least to Dijkstra, Floyd and Hoare in the late
  60's and early 70's} and originally implemented in the Eiffel
language.

In this preliminary chapter we introduce some definitions and
vocabulary, and discuss generalities about this specification language.

Some features in this document are considered ``experimental'',
meaning that their syntax and semantics is still under discussion. We
also consider them as ``advanced features'' which are not supposed to
be useful for a basic use of that specification language.

Chapter~\ref{chap:base} presents the specification language itself.

Chapter~\ref{chap:lib} presents additional informations about
\emph{libraries} of specifications.


\section{Glossary}

An expression is called \emph{pure} if it has no side-effect.

A \emph{contract} is a specification of a function, consisting of the combination of a pre-condition and a collection of \emph{behaviors}.

A \emph{behavior} is mainly a post-condition, possibly associated with
an \emph{frame clause} (called also an \emph{assigns clause} in the
following) specifying side-effects, and possible an assumption on the
pre-state om which the function is called.


\remark{Claude}{ce document doit etre illustr� par des
exemples. Toute construction qui ne serait pas illustr�e par un
exemple sera non-retenue.}

\remark{Patrick}{utiliser en priorit� les constructions de JML,
ensuite les g�n�raliser lorsque cela a un sens, et en dernier recours,
en cr�er de nouvelles. Il y a peut-�tre des constructions
du C++ ou C\# � utiliser ou � g�n�raliser.
Idem avec les extensions de GCC.}

\section{Annotations}

In this document, we consider that annotations are given as comments
written directly into C source files, so
that source files remain compilable\footnote{Other means of attaching
  annotations to source files, without modifying them, is left to user
  tools}. Those comments must start by \verb|/*@| or \verb|//@|.

\subsection{Kinds of annotations}

\begin{itemize}
\item Global annotations:
  \begin{itemize}
  \item Function contract. Such an annotation is inserted just before
    the declaration or the definition of a function.
    See section~\ref{sec:fn-behavior}.


  \item global invariant. This is allowed at the level of global declarations
    See section~\ref{sec:invariants}.

  \item type invariant. This allows both structure or union
    invariants, and invariants on type names introduced by \typedef.
    See section~\ref{sec:invariants}.

  \item logical specifications: logic type introduction, introduction
    or definition of logical function or predicates, axioms. Such an
    annotation is placed at the global declarations level.

  \end{itemize}

\item Statement annotations:
  \begin{itemize}
  \item \assert,\footnote{Remark about \assume clauses: current
      discussion is that it is not considered as an element of
      specification, so not present here. This should be part of proof
      management done by tools}
    logical label. These are allowed everywhere a C label
    is allowed.
    \remark{Patrick}{En C, seules les instructions peuvent �tre
      �tiquett�es. GCC �tend cela aux accolades fermantes des blocs,
      y compris celle fermant le corps de fonction. On peut dire que
      l'on fait de m�me lorsque l'on place des annotations juste avant
      la fermerture d'un bloc.}
    \remark{Patrick}{permettre de poser une assertion sur
      la sortie de fonction quelque soit cette sortie, et d'y
      parler des param�tres, et �ventuellement des variables locales
      d�clar�es en d�but de fonction.}

  \item loop annotation (invariant, variant, assign clauses) are
    allowed immediately before loop statements: \For, \While,
    \Do\ldots \While

  \item statement contract. Very similar to a function contract.
    Semantical condition must be checked (normal termination only, no
    goto going inside, no goto going outside) \remark{Patrick}{as-t'on
      droit au \old dans le \ensures de cette annotation~? Oui, pour
      refer a l'etat avant le statement consider\'er}

  \item ghost brackets for enclosing blocks
    \remark{Patrick}{on a besoin de ``ghost brackets'' pour
      �crire du ``ghost code''. Dans ce cas on n'en fait pas un point
      � part enti�re ici, on en parlera dans la section~\ref{sec:ghost}.
      Par contre, on a besoin de ``logical brackets'' afin de cr�er un
      ``logical statement'' auquel on d�sire associer un ``statement
      behavior''. Faut'il en parler ici, comme on a parl� des
      ``logical label'' plus haut~?}

  \item ghost code. See section~\ref{sec:ghost}

  \end{itemize}

\item Attribute annotations:
  \remark{Patrick}{annotation permettant d'ajouter des attributs
  partout o� l'on peut �crire les qualifications habituelles du
  langage C (const, volatile, restrict) pour les outils qui r�alisent
  de une v�rification et/ou inf�rence de type comme dans
  \texttt{<http://www.cs.umd.edu/\~~jfoster/cqual>}, ou lors de l'inf�rence de
  zones seravnt � identifier un mod�le m�moire. Cela permet �galement de
  proposer des types logiques pour les variables de la logique
  associ�es � des variables du C qui soient diff�rents des types
  propos�s par d�faut.}

\end{itemize}

\subsection{Parsing annotations in practice}

In JML, parsing is done by just ignoring \verb|//@|, \verb|/*@| and \verb|*/| and the level of lexing. This technique can completely modify the semantics of the C code, for example:
\begin{c}
return x /*@ +1 */ ;
\end{c}

In our language this has to be forbidden. Technically, current
implementation of Frama-C isolate the comments in a first step of
syntax analysis, and then parses a second time. Nevertheless, the
grammar and the corresponding parser must be carefully designed to
avoid interaction of annotations with the code. For example, in such a
code:
\begin{c}
if (c) //@ assert P;
   c=1;
\end{c}
the statement \verb|c=1| must be understood as the \texttt{then} branch of the
\texttt{if}. This is ensured by the grammar below, saying that \verb|assert|
annotations are not statement themselves, but attached to the
statement that follows, like C labels.

\subsection{About preprocessing}

This document considers C source \emph{after} preprocessing. Tools must
decide what to do for preprocessing phase: what to do with
annotations, should macro substitution be performed or not, etc.

\subsection{About logical expressions}

Logical expressions are pure C expressions, without function calls.
Semantics is given by the classical 2-valued logic. Additional
constructs exist, given in Section~\ref{sec:expressions}

\subsection{About keywords}

Additional keywords of the specification language start with a
backslash, if they are used in position of a term or a predicate. Otherwise
they do not start with a backslash, like \ensures{}.



%%% Local Variables:
%%% mode: latex
%%% TeX-master: "main"
%%% End:
