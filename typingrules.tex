\section{Typing rules}
\label{sec:typingrules}

Disclaimer: this section is unfinished, it is left here just to give an idea of what it will look like in the final document.

\subsection{Rules for terms}

Integer promotion:
\[
\frac{\Gamma \vdash e : \tau}{\Gamma \vdash e : \integer}
\]
if $\tau$ is any C integer type \verb|char|, \verb|short|, \verb|int|, or \verb|long|, whatever attribute they have, in particular signed or unsigned

Variables:
\[
\frac{}{\Gamma \vdash id : \tau} \mbox{ if $id:\tau\in\Gamma$}
\]

Unary integer operations:
\[
\frac{\Gamma \vdash t : \integer}{\Gamma \vdash op~t : \integer} \mbox{ if $op\in \{+,-,\sim\}$}
\]

Boolean negation:
\[
\frac{\Gamma \vdash t : \bool}{\Gamma \vdash !~t : \bool}
\]

Pointer dereferencing:
\[
\frac{\Gamma \vdash t : \tau*}{\Gamma \vdash *t : \tau}
\]

Address operator:
\[
\frac{\Gamma \vdash t : \tau}{\Gamma \vdash \&t : \tau*}
\]

Binary
\[
\frac{\Gamma \vdash t_1 : \integer\qquad\Gamma \vdash t_2 : \integer}{\Gamma \vdash t_1~op~t_2 : \integer} \mbox{ if $op\in \{+,-,*,/,\%\}$}
\]
\[
\frac{\Gamma \vdash t_1 : \real\qquad\Gamma \vdash t_2 : \real}{\Gamma \vdash t_1~op~t_2 : \real} \mbox{ if $op\in \{+,-,*,/\}$}
\]
\[
\frac{\Gamma \vdash t_1 : \integer\qquad\Gamma \vdash t_2 : \integer}{\Gamma \vdash t_1~op~t_2 : \bool} \mbox{ if $op\in \{==,!=,<=,<,>=,>\}$}
\]
\[
\frac{\Gamma \vdash t_1 : \real\qquad\Gamma \vdash t_2 : \real}{\Gamma \vdash t_1~op~t_2 : \bool} \mbox{ if $op\in \{==,!=,<=,<,>=,>\}$}
\]
\[
\frac{\Gamma \vdash t_1 : \tau*\qquad\Gamma \vdash t_2 : \tau*}{\Gamma \vdash t_1~op~t_2 : \bool} \mbox{ if $op\in \{==,!=,<=,<,>=,>\}$}
\]

(to be continued)


\subsection{Typing rules for sets}
%VP: A-t-on besoin d'un logic environment?
We consider the typing judgement $\Gamma,\Lambda \vdash s : \tau,b$
meaning that $s$ is a set of terms of type $\tau$, which is moreover a
set of locations if the boolean $b$ is true.
$\Gamma$ is the C environment and $\Lambda$ is the logic environment.

Rules:
\[
\frac{}{\Gamma,\Lambda \vdash id : \tau,true} \mbox{ if $id:\tau \in \Gamma$}
\]
\[
\frac{}{\Gamma,\Lambda \vdash id : \tau,true} \mbox{ if $id:\tau \in \Lambda$}
\]
\[
\frac{\Gamma,\Lambda\vdash s:\tau*,b}{\Gamma,\Lambda \vdash *s: \tau,true}
\]
%VP: cette regle ne veut rien dire.
%CM maintenant si, si l'environnement donne les types des champs
\[
\frac{id:\tau \quad s: set<struct~S*>}{\vdash s->id : set<\tau>}
\]
\[
\frac{\Gamma,b\cup \Lambda \vdash e: tset \tau}
{\Gamma,\Lambda\vdash \{ e \mid b ; P \} : tset \tau }
\]
\[
\frac{\Gamma,\Lambda\vdash e_1:\tau,b \quad \Gamma,\Lambda\vdash e_2:\tau,b}
{\Gamma,\Lambda\vdash e_1,e_2: \tau,b}
\]

%%% Local Variables:
%%% mode: latex
%%% TeX-PDF-mode: t
%%% TeX-master: "main"
%%% End:
