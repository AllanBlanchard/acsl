\section{Attributes [C++]}
\label{sec:attributes}

\experimental

\lang allows decorating declarations with \textit{attributes}. 
These can be thought of as supplying small amounts of specification information.

\subsection{[[noreturn]]}
\label{sec:noreturn}

The \lstinline|[[noreturn]]| attribute on a C++ function declaration means
that the function never returns: if it terminates at all, it always throws 
exceptions or exits abruptly. Consequently, the only appropriate \lstinline|ensures| clause
is \lstinline|ensures \false;|.

Thus
\begin{itemize}
\item It is a syntactic error if a function has both a \lstinline|[[noreturn]]|
attribute and an \lstinline|ensures| clause whose predicate is not the 
boolean literal \lstinline|\false|.
\item If a function with an \lstinline|[[noreturn]]| attribute has a behavior
that contains no \lstinline|ensures| clause, the default \lstinline|ensures|
clause is \lstinline|ensures \false;| (rather than the usual \lstinline|ensures \true;|).
\end{itemize}

\subsection{[[noexcept]]}
\label{sec:nexcept}

\TODO{Check on this syntax}

The \lstinline|[[noexcept]]| attribute on a C++ function declaration means
that the function never throws an exception: if it terminates at all, it always terminates normally. Consequently, the only appropriate \lstinline|throws| clause
is \lstinline|throws \false;|.

Thus
\begin{itemize}
	\item It is a syntactic error if a function has both a \lstinline|[[noexcept]]|
	attribute and a \lstinline|throws| clause whose predicate is not the 
	boolean literal \lstinline|\false|.
	\item If a function with a \lstinline|[[noexcept]]| attribute has a behavior
	that contains no \lstinline|throws| clause, the default \lstinline|throws|
	clause is \lstinline|throws \false;| (rather than the usual \lstinline|throws \true;|).
\end{itemize}

