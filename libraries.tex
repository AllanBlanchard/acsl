\chapter{Libraries}
\label{chap:lib}

This chapter is devoted to librairies of specification, built upon the ACSL specification language.

Section~\ref{sec:jessie} describes additional predicates propose by the Jessie plugin of Frama-C, to propose a slightly higher level of annotation.


%\section{Frama-C base library}
%
%\begin{itemize}
%\item \comparable{}: checks whether two pointers are comparable
%  as defined in the ANSI standard.
%  \[
%  \comparable{} : \verb|'a *| \ra \verb|'b *| \ra \boolean
%  \]
%\end{itemize}

\section{Jessie library: logical adressing of memory blocks}
\label{sec:jessie}

%Jessie is a plugin of the Frama-C platform, which connects to the
%corresponding jessie tool of the Why platform~\cite{filliatre07cav}
%for deductive verification of behavioral properties of programs.

The Jessie library is a collection of logic specifications whose
semantics is well-defined only on source codes free from
architecture-dependent features. In particular it is currently
incompatible with pointer casts or unions (although there is ongoing
work to support some of them~\cite{moy07ccpp}). As a consequence a
valid pointer of some type $\tau*$ necessarily points to a memory
block which contains values of type $\tau$.

\begin{remark}{DD5}

Dire que ces limitations ne concernent que cette librairie et ce plugin

\end{remark}


\subsection{Abstract level of pointer validity}

The Jessie plugin currently assumes the input source code free from
architecture-dependent features. In particular it currently completely
disallows pointer casts or unions (although there is ongoing work to
support some of them~\cite{moy07ccpp}). As a consequence a valid
pointer of some type $\tau*$ necessarily points to a memory block
which contains values of type $\tau$. To model that, the jessie library introduce new logic functions:
\begin{flushleft}
\integer ~ \offsetmin('a *p); \\
\integer ~ \offsetmax('a *p);
\end{flushleft}

\begin{itemize}
\item $\offsetmin(p)$ is the minimum integer $i$ such that $(p+i)$ is a
  valid pointer.

\item $\offsetmax(p)$: the maximum integer $i$ such that $(p+i)$ is a
  valid pointer
\end{itemize}
The following properties hold:
\begin{eqnarray*}
\offsetmin(p+i) &=& \offsetmin(p)-i \\
\offsetmax(p+i) &=& \offsetmax(p)-i
\end{eqnarray*}
It also introduce syntactic sugar:
\begin{eqnarray*}
\validrange(p,i,j) &:=& \offsetmin(p) <= i \land \offsetmax(p) >= j 
\end{eqnarray*}
and the ACSL built-in predicate $\valid(p)$ is now equivalent to
$\validrange(p,0,0)$.

\subsection{Strings}

\experimental

The predicate 
\[
\integer~\strlen(\verb|char*|p)
\]
denotes the length of a 0-terminated C string. It is total function,
whose value is non-negative if and only if the poiner in argument is
really a string.

\begin{example}
  Here is a contract for the \verb|strcpy| function:
  \input{strcpyspec.pp}

\end{example}

\subsection{Field structures}

 "offsetof" "(" ... ")" ; \experimental

 "alignof" "(" C-type-expr ")" ; \experimental

\section{Memory leaks}

\experimental

Verification of absence of memory leak is outside the scope of the
specification language. On the other hand, various models could be set
up, using for example ghost variables.

\section{Libraries of logic specifications}
\label{sec:speclibraries}

A standard library is provided, in the spirit of the List module of
Section~\ref{sec:specmodules}

%%% Local Variables:
%%% mode: latex
%%% TeX-PDF-mode: t
%%% TeX-master: "main"
%%% End:
