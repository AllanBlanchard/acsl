\section{Automatic type inference [C++]}
\label{sec:auto}
\index{auto}

\subsection{auto}
Beginning with C++11\footnote{Prior to C++11, \lstinline|auto| had a different meaning}, \lang permits designating the type of a variable as
\lstinline|auto|.
 \lstinline|auto| may also be used as part of a type expression, as in \lstinline|auto*| or \lstinline|const auto[]|. In these cases the compiler performs type inference, producing a compile-time type and
saving the user from having to write out the type.

A second use of \lstinline|auto| is as a simple way of keeping the types
within some code flexible. For example, one might write
\begin{lstlisting}
auto f(int x) {
   auto y = ... computation involving x ...
   auto z = ... more computation using x and y ...
   return z;
}
\end{lstlisting}
Here the return type depends on the type of \lstinline|x| but is not hard-coded into the body of \lstinline|f|. If one changes the declared type of
\lstinline|x| to say \lstinline|double|, all the inferred types change without needing further editing.

The disadvantage of \lstinline|auto| is of course, that although the 
program is still strongly typed, the actual types are harder for the reader to discern.

\lstinline|auto| is also permitted within \NAME text, where the type can be inferred at compile time.
In particular, if the variable has an initializer, the type is readily inferred.
 The inferred type may be a logic type. For example, in
\begin{lstlisting}
//@ logic auto f(integer i) = i + 1;
\end{lstlisting}
the result type can be inferred to be \lstinline|integer|.

Within \NAME specification text, the \lstinline|auto| keyword can be used in these locations:
\begin{itemize}
	\item as the result type in a logic function declaration
	\item as the variable type in a ghost declaration with initialization
\end{itemize}

\subsection{decltype}

\NAME annotations may also use the \lstinline|decltype| feature of \lang. \lstinline|decltype| takes a single argument, either a type or an expression. The keyword and its argument are
replaced by the type of the expression. The effect is precisely as in \lang, and the reader should refer to the \lang standard or explanations for details.


