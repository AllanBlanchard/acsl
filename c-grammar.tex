\section{C grammar elements}
\label{sec:cgrammar}

This appendix chapter summarizes the elements of the \lang grammar that are used by \NAME.

\subsection{Identifiers}

\begin{itemize}
  \item  An \textit{id} is a sequence of alphanumeric characters and underscores beginning with a non-digit:  \lstinline| [a-zA-Z_][a-zA-Z_0-9]* |.
\end{itemize}

\subsection{Literals}

Numeric, character and string literals are adopted from \lang without modification:
\begin{itemize}
  \item An \textit{integer literal} is a digit sequence with optional radix and bit-size indicators: \\ \lstinline! [+-]?(0 | [1-9][0-9]* | 0[0-7]+ | 0x[0-9A-Za-z]+) !
  \item A \textit{real literal} has the following form:
 \begin{lstlisting}
   [+-]? ([0-9]+)? (\.[0-9]*)? ([eE][+-]?(0-9)+)? ([fFlL])?
   \end{lstlisting}
where there must be either an initial digit sequence or post-decimal point digit sequence and either a decimal point or an exponent.\footnote{In C++17 exponents beginning with \lstinline|p| or \lstinline|P| and hex digit sequences with leading 0x are permitted.}
  \item A \textit{character literal} is a single character or a backslash followed by a single character enclosed in single quotes, optionally 
  preceded by the \lstinline|L| character to denote a wide character.
  \item A \textit{string literal} is a sequence of printable characters or escape sequences enclosed in double quotes: TODO
\end{itemize}


\subsection{C Type Expressions}

\begin{figure}[htp]
  \begin{cadre}
    \begin{syntax}
C-type-expr ::= C-specifier-qualifier+ C-abstract-declarator?
\
C-type-name ::= C-declaration-specifier+
\
C-specifier-qualifier ::= C-type-specifier | C-type-qualifier
\
C-type-qualifier ::= "const" | "volatile"
\
C-type-specifier ::= "void" ;
                   | "char" ;
                   | "short" ;
                   | "int" ;
                   | "long" ;
                   | "float" ;
                   | "double" ;
                   | "signed" ;
                   | "unsigned" ;
                   | ( "struct" | "union" | "enum" ) ident [ACSL does not permit declaring a new type within the type-specifier] ;
                   | ident
\
C-abstract-declarator ::= C-pointer ;
| C-pointer C-direct-abstract-declarator ;
| C-direct-abstract-declarator
\
C-pointer ::= ( "*" C-type-qualifier* )+
\
C-direct-abstract-declarator ::= "(" C-abstract-declarator ")" ; 
| C-direct-abstract-declarator? "[" C-constant-expression "]" ;
| C-direct-abstract-declarator? "(" C-parameter-type-list? ")"
\
C-parameter-type-list ::= C-parameter-declaration ("," C-parameter-declaration )+
\
C-parameter-declaration ::= C-declaration-specifier+ C-declarator ;
                          | C-declaration-specifier+ C-abstract-declarator ;
                          | C-declaration-specifier+ ;
\
C-declaration-specifier ::= C-type-specifier | C-type-qualifier
\
C-declarator ::= C-pointer? C-direct-declarator
\
C-direct-declarator ::= ident ;
                      | "(" C-declarator ")" ;
                      | C-direct-declarator "[" C-constant-expression? "]" ;
                      | C-direct-declarator "(" C-parameter-type-list ")" ;
                      | C-direct-declarator "(" ident* ")"
\
C-constant-expression ::=  ... [An expression formed from constant literals]
\end{syntax}

  \end{cadre}
  \caption{The grammar of C type expressions, from the C standard}
\label{fig:gram:ctype}
\end{figure}

\ifCPP{\TODO{Review the figure on C type expressions}}

