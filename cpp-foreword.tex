\thispagestyle{plain}
\chapter*{Foreword [C++]}

This document describes version \acslppversion{}
of the ANSI/ISO \lang Specification Language (\NAME). 
The language features will evolve in the future. 
In particular, some features in this document
are considered \emph{experimental}, meaning that their syntax and
semantics is not yet fixed.  
These features are marked with
\experimental.  
They must also be considered advanced features,
which are not needed for basic use of this
specification language.

\section*{Acknowledgements --- \NAME}

This language design and document incorporate nearly all the 
ACSL language and documentation; many portions of the text
are explicitly shared. 
Some sections are nearly verbatim 
duplicates of the corresponding ACSL material, with light edits
to reflect the use in \NAME~ as well. 
Other sections, marked \texttt{[C++]}, are new to \NAME.
\ifCPP{
  In both cases, \NAME specificities with respect to ACSL will appear
  in dark green, like this sentence.
}

Thus we acknowledge the extensive
set of collaborators and commenters on ACSL that are listed
in the companion ACSL documentation and replicated below.

The design of \NAME~also draws from the specification
languages of other object-oriented programming languages,
such as Java \cite{leavens00jml}.

The following have contributed explicitly to the formation of
this document and the design of \NAME~itself.
We gratefully thank all the people who contributed to this document:

David Cok,
Patrick Baudin,
Virgile Prevosto,
Franck Vedrine.

\section*{Acknowledgements --- ACSL}
There are many contributors to ACSL, including principally the following
(as listed in the ACSL Reference Manual):
Sylvie Boldo,
David Cok,
Jean-Louis Cola\c{c}o,
Pierre Cr\'egut,
David Delmas,
Catherine Dubois,
St\'ephane Duprat,
Arnaud Gotlieb,
Philippe Herrmann,
Thierry Hubert,
Andr\'e Maroneze,
Dillon Pariente,
Pierre Rousseau,
Julien Signoles,
Jean Souyris,
Asma Tafat.


\section*{Funding}
The writing of this document describing \NAME is done in the context of
project VESSEDIA,
which  received funding from the European Union's 2020
Research and Innovation Program under grant agreement
No. 731453.
