\section{Templates [C++]}
\label{sec:templates}
\index{templates}
\subsection{Class templates}

\lang includes syntax for parameterized template classes and functions.  \NAME constructs can occur within template classes, just as they can appear within non-template classes. However, note that \NAME (like \acsl) has its own syntax for declaring parameterized (polymorphic) types, so no additional 
capabilities are needed within \NAME to allow
parameterized logic types, functions, or predicates.

Naively, the consequence of an \NAME declaration being within a template is that some additional names are defined in the scope in which the \NAME declaration is declared. In particular, most commonly there are additional type names, but there can also be values of other types. Any \NAME contracts within the template can use the template parameters, just like any \lang construct can. 

However, templates have a complication for both \lang and \NAME: until template parameters are given actual values (that is, until the template is instantiated) it cannot always be determined whether the code or specifications within the template are well-formed. This is because the declared names and the types of names and operations that are dependent on the template parameters cannot be resolved until the actual template parameters are known.  If \lstinline|T| is a template parameter, \lstinline|T::a| is not known to be syntactically valid, or, if it is, what its type is, until the value of \lstinline|T| itself is known.

Therefore a library template cannot be fully verified as a parameterized class; only verifications of instantiations are possible.

\subsection{Specifications of templates [C++]}

Even if a parameterized template cannot be verified until it is instantiated, the template text must still contain its specification. That specification must be written using names and operations that are in scope within the template specification, even if they are instantiated along with the template itself.

In the example below, the template methods \lstinline|length| and \lstinline|intsum| of an implementation of a singly-linked list each have corresponding logic functions. The logic functions can use the data fields of the class. The logic function \lstinline|intsum| is total, but only has a known value if the \lstinline|static_cast| is valid, that is, if \lstinline|value| can be legitimately cast to an \lstinline|int|. In the post-condition of the \lang function \lstinline|intsum|, the condition giving the \lstinline|\result| is conditioned on the type being \lstinline|int|.
\begin{example}
 
\begin{listing-nonumber}
#include <type_traits>

template<class T> class list {
  list<T>* next;
  T value;
  
public:

  //@ int length() = 1 + (next == nullptr ? 0: next.length());

  //@ ensures \result == length();
  int length();
  
  //@ int intsum() =  static_cast<int>(value) + (next == nullptr ? 0 : next.intsum());
  
  //@ ensures std::is_same_type<T,int>::value ==> \result == intsum();
  int intsum();
  
}
\end{listing-nonumber}

\end{example} 

\subsection{Function templates [C++]}

\lang also allows declaring function templates, either as global functions or as members of aggregate classes. The corresponding capability is already present in \NAME as logic predicates and functions that have polymorphic type arguments.

In the following example, the templated function \lstinline|larger| requires that the template type argument have an operator \lstinline|>| defined; if so, it returns the larger of two items according to that comparison. The difficulty with specifying such a template is that we need the comparable generality in writing \NAME annotations. The example shows a vbehavior case conditioned onn the type parameter. One might also write logic functions representing the comparison for cases that do not have a logic interpretation of \lstinline|>|.

\begin{example}

\begin{listing-nonumber}
#include <type_traits>
/*@ behavior int :
       assume std::is_same_type<M,int>::value;
       ensures \result == (a > b ? a : b);
 */
template <class M>
M larger(M a, M b) {
  return a > b ? a : b;
}
\end{listing-nonumber}

\end{example}

\ifImpl{Status: \NAME specifications within template classes or attached to template functions are not yet implemented.}

\TODO{Check specs on template specializations}
