\section{Templates}
\label{sec:tempaltes}
\index{templates}
\subsection{Class templates}

\lang{} includes syntax for parameterized template classes and functions. \NAME{} does not permit the declaration or definition of classes (or other aggregates) or templates  within \NAME{} annotations. However, \NAME{} constructs can occur within template classes. Note that \NAME{} (like ACSL) does have its own syntax for declaring parameterized (polymorphic) types.

Naively, the consequence of an \NAME{} declaration being within a template is that some additional names are defined in the scope in which the \NAME{} declaration is declared. In particular, most commonly there are additional type names, but there can also be values of other types. Any \NAME{} contracts within the template can use the template parameters, just like any \lang{} construct can. 

However, templates have a complication for both \lang{} and \NAME{}: until template parameters are given actual values (that is, until the template is instantiated) it cannot always be determined whether the code or specifications within the template are well-formed. This is because the declared names and the types of names and operations that are dependent on the template parameters cannot be resolved until the actual template parameters is known.  If \lstinline|T| is a template parameter, \lstinline|T::a| is not known to be syntactically valid, or, if it is, what its type is, until the value of \lstinline|T| itself is known.

Therefore a library template cannot be fully verified as a parameterized class; only verifications of instantiations are possible.

\section{Specifications of templates}

Even if a parameterized template cannot be verified until it is instantiated, the template text must still contain its specification. That specification must be written using names and operations that are in scope within the template specification, even if they are instantiated along with the template itself.

\section{Function templates}

\lang{} also allows declaring function templates, either as global functions or as members of aggregate classes. The corresponding capability is already present in \NAME{} as logic predicates and functions that have polymorphic type arguments.