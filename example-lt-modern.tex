Let us consider the following example:
\begin{listing-nonumber}
  int f(int a, int b) { return a < b; }
\end{listing-nonumber}
\begin{itemize}
\item the obvious postcondition \lstinline[style=c]|\result == a < b|
  is not the right one because it is actually a shortcut for
  \lstinline[style=c]|\result == a && a < b|.
\item adding parentheses results in a correct post-condition
  \lstinline[style=c]|\result == (a < b)|. 
Note however that there is an implicit
  conversion (see Sec.~\ref{sec:typing})
  from the \lstinline[style=c]|int| (the type of \lstinline[style=c]|\result|) to
  \lstinline[style=c]|boolean| (the type of \lstinline[style=c]|(a<b)|)
\item an equivalent post-condition, which does not rely on implicit
  conversion, is \lstinline[style=c]|(\result != 0) == (a<b)|. Both pairs of
  parentheses are mandatory.
\item \lstinline[style=c]|\result == (integer)(a<b)| is also acceptable because it compares
  two integers. The cast towards \lstinline[style=c]|integer| enforces
  \lstinline[style=c]|a<b| to be understood as a boolean term. Notice that a cast
  towards \lstinline[style=c]|int| would also be acceptable.
\item \lstinline[style=c]|\result != 0 <==> a < b| is acceptable because it is an
  equivalence between two predicates.
ependencies
\end{itemize}
