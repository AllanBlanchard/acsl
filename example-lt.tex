

Let's consider the following example:
\begin{c}
  int f(int a, int b) { return a < b; }
\end{c}
The following post-conditions are wrong:
\begin{itemize}
\item the obvious post-condition \verb|\result == a < b| is not the
  right one because it is 
  actually a shortcut for \verb|\result == a && a < b|.
\item adding parentheses results in an ill-typed post-condition
  \verb|\result == (a < b)|, 
  because it tests equality of \verb|\result| which has
  type \verb|int| with \verb|a < b| which has type \verb|boolean|.
\item similarly, \verb|\result <==> a < b| is not well-typed, because
  it makes an equivalence between an \verb|int| and a predicate.  
\end{itemize}
The following are correct post-conditions:
\begin{itemize}
\item \verb|\result != 0 <==> a < b| is acceptable because it is an
  equivalence between two predicates.
\item \verb|\result == (integer)(a<b)| is also acceptable because it compares
  two integers. The cast towards \verb|integer| enforces
  \verb|a<b| to be understood as a boolean term. Notice that a cast
  towards \verb|int| would also be acceptable. 
\end{itemize}

