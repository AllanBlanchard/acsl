

Let's consider the following example:
\begin{c}
  int f(int a, int b) { return a < b; }
\end{c}
\begin{itemize}
\item the obvious postcondition \verb|\result == a < b| is not the
  right one because it is
  actually a shortcut for \verb|\result == a && a < b|.
\item adding parentheses results in a correct post-condition
  \verb|\result == (a < b)|. Note however that there is an implicit
  conversion (see Sec.~\ref{sec:typing})
  from the \verb|int| (the type of \verb|\result|) to
  \verb|boolean| (the type of \verb|(a<b)|)
\item an equivalent post-condition, which does not rely on implicit
  conversion, is \verb|(\result != 0) == (a<b)|. Both pairs of
  parentheses are mandatory.
\item \verb|\result == (integer)(a<b)| is also acceptable because it compares
  two integers. The cast towards \verb|integer| enforces
  \verb|a<b| to be understood as a boolean term. Notice that a cast
  towards \verb|int| would also be acceptable.
\item \verb|\result != 0 <==> a < b| is acceptable because it is an
  equivalence between two predicates.
\end{itemize}
