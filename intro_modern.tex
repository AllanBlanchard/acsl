%; whizzy-master "main.tex"
\chapter{Introduction}

This document is a reference manual for
\ifthenelse{\boolean{PrintImplementationRq}}%
           {the \NAME{} implementation provided by the
             \mbox{\textsc{Frama-C}} %no cesure
            framework~\cite{frama-c}.}%
           {ACSL.}
\NAME{} is an acronym for ``ANSI/ISO \lang{} 
Specification Language''. 
This is a Behavioral Interface Specification Language (BISL)
\ifthenelse{\boolean{PrintImplementationRq}}%
           {for}%
           {implemented in the \textsc{Frama-C} framework. It aims at}
specifying behavioral properties of \lang{}
source code.

\ifthenelse{\boolean{PrintImplementationRq}}{%
Not all of the features mentioned in this document are currently
implemented in the Frama-C kernel. Unimplemented features are signaled
as in the following line:
\begin{quote}
\notimplemented[Additional remarks on the feature may appear in a footnote.]%
{This feature is not currently supported by Frama-C}
\end{quote}

As a summary, the features that are not currently implemented in
Frama-C include in particular:
\begin{itemize}
\item some built-in predicates and logical functions;
\item definition of logical types (section~\ref{sec:logicspec});
\item specification modules (section~\ref{sec:specmodules});
\item model variables (section~\ref{ex:model});
\item only basic support for ghost code is provided (section~\ref{sec:ghost});
\item verification of non interference of ghost code
  (p.~\pageref{sec:semantics-ghost-code});
\item specification of volatile variables
  (section~\ref{sec:volatile-variables});
\end{itemize}
}%

The main inspiration for this language comes from the
specification language of the \textsc{Caduceus}
tool~\cite{filliatre04icfem,filliatre07cav} for deductive verification
of behavioral properties of C programs. The specification language of
Caduceus is itself inspired from the
\emph{Java Modeling Language} (JML~\cite{leavens00jml}) which aims at
similar goals for Java source code: indeed it aims both at
\emph{runtime assertion checking} and \emph{static verification} using
the \textsc{OpenJML} tool~\cite{Cok-2011-OpenJML,Cok-2014-OpenJML}, where we aim at
\emph{static verification} and \emph{deductive verification} (see
Appendix~\ref{sec:comp-jml} for a detailed comparison between \NAME~and
JML).

Going back further in history, the JML design was guided by the general
\emph{design-by-contract} principle proposed by Bertrand Meyer,
originally implemented in
the \textsc{Eiffel} language;
he took his inspiration from the concepts of preconditions and
postconditions on a routine, going back at least to Dijkstra, Floyd and
Hoare in the late 60's and early 70's.

In this document, we assume that the reader has a good knowledge of
the ISO C programming language~\cite{KR88,standardc99} 
\ifthenelse{\isCPP}{and the ISO C++ programming language~\cite{C++-2011}}{}.

\section{Organization of this document}

In this preliminary chapter we introduce some definitions and
vocabulary, and discuss generalities about this specification
language.  Chapter~\ref{chap:base} presents the specification language
itself.  Chapter~\ref{chap:lib} presents additional information about
\emph{libraries} of specifications. The appendices provide
specific formal type-checking rules for \NAME{} annotations, 
the relation between \NAME{} and JML, and specification templates.\marginpar{specicfication templates?}
A detailed table of
contents is given on page~\pageref{chap:contents}.
A glossary is given in Appendix~\ref{sec:glossary}.

\section{Generalities about Annotations}\label{sec:gener-about-annot}

In this document, we consider that specifications are given as
annotations in comments written directly in \lang{} source files, so that
source files remain compilable. Those comments must start with
\verb|/*@| or \verb|//@| and end as usual in \lang.

In some contexts, it is not possible to modify the source code.
It is strongly recommended that a tool that implements
\lang{} specifications provide technical means to store annotations
separately from the source. It is not the purpose of this document
to describe such means.  Nevertheless, some of the specifications,
namely those at a global level, can be given in separate files:
logical specifications can be imported (see
Section~\ref{sec:specmodules}) and a function contract can be attached
to a copy of the function profile (see
Section~\ref{sec:multiplecontracts}).

\subsection{Kinds of annotations}

\begin{itemize}
\item Global annotations:
  \begin{itemize}
  \item \emph{function contract}\,: such an annotation is inserted just before
    the declaration or the definition of a function.
    See section~\ref{sec:fn-behavior}.

  \item \emph{global invariant}\,: this is allowed at the level of
    global declarations. See section~\ref{sec:invariants}.

  \item \emph{type invariant}\,: this allows declaring  structure invariants,
     union invariants, 
     \ifthenelse{\isCPP}{class invariants,}{}
     and invariants on type names introduced by
    \typedef.  See section~\ref{sec:invariants}.

  \item \emph{logic specifications}\,: definitions of logic functions
    or predicates, lemmas, axiomatizations by declaration of new logic
    types, logic functions, predicates with axioms they satisfy. Such an
    annotation is placed at the level of global declarations. 
    See section~\ref{sec:logicspec}

  \end{itemize}

\item Statement annotations:
  \begin{itemize}
  \item \emph{assertion}\,: these are allowed
    everywhere a C label is allowed, or just before a
    block closing brace. See section~\ref{sec:assertions}.

  \item \emph{loop annotation} (invariant, variant, assign clauses): is
    allowed immediately before a loop statement: \For, \While,
    \Do\ldots \While. See Section~\ref{sec:loop_annot}.

  \item \emph{statement contract}\,: very similar to a function contract, and
    placed before a statement or a block.  Semantic conditions must
    be checked (no goto going inside, no goto
    going outside).  See Section~\ref{sec:statement_contract}.

  \item \emph{ghost code}\,: regular C code, only visible from the
    specifications, that is only allowed to modify ghost
    variables. See section~\ref{sec:ghost}. This includes ghost braces
    for enclosing blocks.

  \end{itemize}

\end{itemize}

\subsection{Parsing annotations in practice}

In JML, parsing is done by simply ignoring \verb|//@|, \verb|/*@| and
\verb|*/| at the lexical analysis level. This technique could modify the
semantics of the code, for example:
\begin{listing}{1}
return x /*@ +1 */ ;
\end{listing}
In our language, this is forbidden. Technically, the current
implementation of Frama-C isolates the comments in a first step of
syntax analysis, and then parses a second time. Nevertheless, the
grammar and the corresponding parser must be carefully designed to
avoid interaction of annotations with the code. For example, in code such as
\begin{listing}{1}
  if (c) //@ assert P;
     c=1;
\end{listing}
the statement \lstinline|c=1| must be understood as the
branch of the \texttt{if}. This is ensured by the ACSL grammar,
which states that \lstinline|assert| annotations are not statements themselves,
but attached to the statement that follows, like C labels.


\subsection{About preprocessing}

This document considers \lang{} source \emph{after} preprocessing.
Tools must decide how they handle preprocessing (what to do with
annotations, whether macro substitution should be performed, etc.)

\subsection{About keywords}

Additional keywords of the specification language start with a
backslash, if they are used in the position of a term or a predicate
(which are defined later in the document).  Otherwise they do not start
with a backslash (like \ensures{}) and they remain valid identifiers.

\section{Notations for grammars}

In this document, grammar rules are given in BNF form. In the grammar
rules, we use the extra notations $e^*$ to denote repetition of zero, one
or more occurrences of $e$, $e^+$ for repetition of one or more
occurrences of $e$, and $e^?$ for zero or one occurrence of $e$.  For
the sake of simplicity, we only describe annotations in the usual
\lstinline|/*@ ... */| style of comments. One-line annotations
in \lstinline|//@| comments are similar. Note however that two consecutive
comments, regardless of their style, are considered as two independent
annotations. In particular, it is not possible to split a multi-line
annotation into several \lstinline|//@| comments.

%%% Local Variables:
%%% mode: latex
%%% TeX-PDF-mode: t
%%% TeX-master: "main"
%%% End:
