%; whizzy-master "main.tex"
\chapter{Introduction}

This document is a reference manual for
\ifthenelse{\boolean{PrintImplementationRq}}%
{the ACSL implementation provided by the Frama-C
  framework~\cite{frama-c}.}%
{ACSL.}
ACSL is an acronym for ``ANSI/ISO C
Specification Language''. This is a Behavioral Interface Specification
Language (BISL) implemented in the \textsc{Frama-C} framework.
It aims at specifying behavioral properties of C
source code. The main inspiration for this language comes from the
specification language of the \textsc{Caduceus}
tool~\cite{filliatre04icfem,filliatre07cav} for deductive verification
of behavioral properties of C programs. The specification language of
Caduceus is itself inspired from the
\emph{Java Modeling Language} (JML~\cite{leavens00jml}) which aims at
similar goals for Java source code: indeed it aims both at
\emph{runtime assertion checking} and \emph{static verification} using
the \textsc{ESC/Java2} tool~\cite{ESCJava2}, where we aim at
\emph{static verification} and \emph{deductive verification} (see
Appendix~\ref{sec:comp-jml} for a detailed comparison between ACSL and
JML).

Going back further in history, JML design was guided by the general
\emph{design-by-contract} principle proposed by Bertrand Meyer, who
took his own inspiration from the concepts of preconditions and
postconditions on a routine, going back at least to Dijkstra, Floyd and
Hoare in the late 60's and early 70's, and originally implemented in
the \textsc{Eiffel} language.

In this document, we assume that the reader has a good knowledge of
the ISO C programming language~\cite{KR88,standardc99}.

\section{Organization of this document}

In this preliminary chapter we introduce some definitions and
vocabulary, and discuss generalities about this specification
language.  Chapter~\ref{chap:base} presents the specification language
itself.  Chapter~\ref{chap:lib} presents additional information about
\emph{libraries} of specifications. Appendix~\ref{chap:appendix} provides
specific hindsight over type-checking ACSL annotations, 
the relation between ACSL and JML, and specification templates.
A detailed table of
contents is given on page~\pageref{chap:contents}.
A glossary is given in Appendix~\ref{sec:glossary}.

\ifthenelse{\boolean{PrintImplementationRq}}{%
Not all of the features mentioned in this document are currently
implemented in the Frama-C kernel. Those who aren't yet are signaled
as in the following line:
\begin{quote}
\notimplemented[Additional remarks on the feature may appear as footnote]%
{This feature is not currently supported by Frama-C}
\end{quote}

As a summary, the features that are not currently implemented into
Frama-C include in particular:
\begin{itemize}
\item some built-in predicates and logical functions;
\item abrupt termination clauses in statement contracts
  (section~\ref{sec:statement_contract});
\item definition of logical types (section~\ref{sec:logicspec});
\item specification modules (section~\ref{sec:specmodules});
\item model variables and Model fields (section~\ref{ex:model});
\item only basic support for ghost code is provided (section~\ref{sec:ghost});
\item verification of non interference of ghost code
  (p.~\pageref{sec:semantics-ghost-code});
\item specification of volatile variables
  (section~\ref{sec:volatile-variables});
\end{itemize}
}%
{}

\oldremark{Claude}{
ce document doit etre illustré par des
exemples. Toute construction qui ne serait pas illustrée par un
exemple sera non-retenue.
}

\oldremark{Patrick}{
utiliser en priorité les constructions de JML,
ensuite les généraliser lorsque cela a un sens, et en dernier recours,
en créer de nouvelles. Il y a peut-être des constructions
du C++ ou C\# à utiliser ou à généraliser.
Idem avec les extensions de GCC.
}


\section{Generalities about Annotations}\label{sec:gener-about-annot}

In this document, we consider that specifications are given as
annotations in comments written directly in C source files, so that
source files remain compilable. Those comments must start by
\verb|/*@| or \verb|//@| and end as usual in~C.

In some contexts, it is not possible to modify the source code.
It is strongly recommended that a tool which implements
ACSL specifications provides technical means to store annotations
separately from the source. It is not the purpose of this document
to describe such means.  Nevertheless, some of the specifications,
namely those at a global level, can be given in separate files:
logical specifications can be imported (see
Section~\ref{sec:specmodules}) and a function contract can be attached
to a copy of the function profile (see
Section~\ref{sec:multiplecontracts}).


\oldremark{All}{

  Elaborer plus sur ce que peuvent faire les outils pour permettre des
  annotations dans des fichiers a part.

  Mettre a part le cas simple des annotations globales (qui peuvent
  etre dans un .h a part)

}

\oldremark{DP}{

  discuter sur les facons d'annoter les statements sans modifier le code source
  mais en les donnant a part.

  idee :

  . fichier patchs
  . autres

  a discuter en appendice de ce document. (?)

}





\subsection{Kinds of annotations}

\begin{itemize}
\item Global annotations:
  \begin{itemize}
  \item \emph{function contract}\,: such an annotation is inserted just before
    the declaration or the definition of a function.
    See section~\ref{sec:fn-behavior}.

  \item \emph{global invariant}\,: this is allowed at the level of
    global declarations. See section~\ref{sec:invariants}.

  \item \emph{type invariant}\,: this allows to declare both structure
    or union invariants, and invariants on type names introduced by
    \typedef.  See section~\ref{sec:invariants}.

  \item \emph{logic specifications}\,: definitions of logic functions
    or predicates, lemmas, axiomatizations by declaration of new logic
    types, logic functions, predicates with axioms they satisfy. Such an
    annotation is placed at the level of global declarations. 
    See section~\ref{sec:logicspec}

  \end{itemize}

\item Statement annotations:
  \begin{itemize}
  \item \emph{assertion}\,: these are allowed
    everywhere a C label is allowed, or just before a
    block closing brace. See section~\ref{sec:assertions}.
    \oldremark{Claude}{About \assume{} clauses: current
      discussion is that it is not considered as an element of
      specification, so not present here. This should be part of proof
      management done by tools.
    }
    \oldremark{Patrick}{En C, seules les
      instructions peuvent être étiquettées. GCC étend cela aux
      accolades fermantes des blocs, y compris celle fermant le corps
      de fonction. On peut dire que l'on fait de même lorsque l'on
      place des annotations juste avant la fermerture d'un bloc.
    }

  \item \emph{loop annotation} (invariant, variant, assign clauses): is
    allowed immediately before a loop statement: \For, \While,
    \Do\ldots \While. See Section~\ref{sec:loop_annot}.

  \item \emph{statement contract}\,: very similar to a function contract, and
    placed before a statement or a block.  Semantic conditions must
    be checked (no goto going inside, no goto
    going outside).  See Section~\ref{sec:statement_contract}.
    \oldremark{Patrick}{
      as-t'on droit au \old dans le \ensures de
      cette annotation~? Oui, pour refer a l'etat avant le statement
      consider\'e
    }

  \item \emph{ghost code}\,: regular C code, only visible from the
    specifications, that is only allowed to modify ghost
    variables. See section~\ref{sec:ghost}. This includes ghost braces
    for enclosing blocks.

  \end{itemize}

%\item Attribute annotations: \experimental. See
%  Section~\ref{sec:attribute_annot}.

\end{itemize}

\subsection{Parsing annotations in practice}

In JML, parsing is done by simply ignoring \verb|//@|, \verb|/*@| and
\verb|*/| at the lexical analysis level. This technique could modify the
semantics of the code, for example:
\begin{listing}{1}
return x /*@ +1 */ ;
\end{listing}
In our language, this is forbidden. Technically, the current
implementation of Frama-C isolates the comments in a first step of
syntax analysis, and then parses a second time. Nevertheless, the
grammar and the corresponding parser must be carefully designed to
avoid interaction of annotations with the code. For example, in code such as
\begin{listing}{1}
  if (c) //@ assert P;
     c=1;
\end{listing}
the statement \lstinline|c=1| must be understood as the
branch of the \texttt{if}. This is ensured by the ACSL grammar,
which states that \lstinline|assert| annotations are not statements themselves,
but attached to the statement that follows, like C labels.


\subsection{About preprocessing}

This document considers C source \emph{after} preprocessing.
Tools must decide how they handle preprocessing (what to do with
annotations, whether macro substitution should be performed, etc.)

\subsection{About keywords}

Additional keywords of the specification language start with a
backslash, if they are used in the position of a term or a predicate
(which are defined in the following).  Otherwise they do not start
with a backslash (like \ensures{}) and they remain valid identifiers.
\oldremark{Claude}{
cas particulier de 'integer' et 'real'  et 'boolean'
}
\section{Notations for grammars}

In this document, grammar rules are given in BNF form. In grammar
rules, we use extra notations $e^*$ to denote repetition of zero, one
or more occurrences of $e$, $e^+$ for repetition of one or more
occurrences of $e$, and $e^?$ for zero or one occurrence of $e$.  For
the sake of simplicity, we only describe annotations in the usual
\lstinline|/*@ ... */| style of comments. One-line annotations
in \lstinline|//@| comments are alike.


%%% Local Variables:
%%% mode: latex
%%% TeX-PDF-mode: t
%%% TeX-master: "main"
%%% End:
